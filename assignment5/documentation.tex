\documentclass[pdftex, 12pt]{article}

\usepackage[pdftex]{graphicx}
\usepackage{listings}
\usepackage{fancyhdr}
\usepackage{lastpage}
\usepackage{fullpage}
\usepackage{color}
\usepackage{amsmath}
\usepackage[pdftex,bookmarks=true,colorlinks=true,linkcolor=blue]{hyperref}
\usepackage{subfig}
\usepackage{amssymb}


%this sets the line at the header
\setlength{\headheight}{15.2pt}

\newcommand{\HRule}{\rule{\linewidth}{0.5mm}}

% This is all for formatting and making the Table of Contents according to 
% spec. Don't play with it.
\makeatletter
\renewcommand\l@section[2]{%
  \ifnum \c@tocdepth >\z@
    \addpenalty\@secpenalty
    \addvspace{1.0em \@plus\p@}%
    \setlength\@tempdima{1.5em}%
    \begingroup
      \parindent \z@ \rightskip \@pnumwidth
      \parfillskip -\@pnumwidth
      \leavevmode \bfseries
      \advance\leftskip\@tempdima
      \hskip -\leftskip
      #1\nobreak\ 
      \leaders\hbox{$\m@th\mkern \@dotsep mu\hbox{.}\mkern \@dotsep mu$}
     \hfil \nobreak\hb@xt@\@pnumwidth{\hss #2}\par
    \endgroup
  \fi}
\makeatother



\begin{document}

%import the title page
\begin{titlepage}
	\begin{center}

		% Upper part of the page
		\textsc{\LARGE University of Nevada, Reno}\\[.5cm]
		\includegraphics[width=0.15\textwidth]{./logo.png}\\[.5cm]

		\textsc{\large CS 302 | Data Structures } \\[.5cm]

		% Title
		\HRule \\[0.4cm]
		{ \huge \bfseries Assignment \#5}\\[0.4cm]

		\HRule \\[1.5cm]

		% Author and supervisor
		\begin{minipage}{0.4\textwidth}
			\begin{flushleft} \large
				\emph{Students:}\\
				Joshua \textsc{Gleason}\\
				Josiah \textsc{Humphrey}
			\end{flushleft}
		\end{minipage}
		\begin{minipage}{0.4\textwidth}
			\begin{flushright} \large
				\emph{Instructor:} \\
				Dr. George \textsc{Bebis}
			\end{flushright}
		\end{minipage}

		\vfill

		% Bottom of the page
		{\large \today}

	\end{center}

\end{titlepage}


%headers, footers, and table of contents
\pagestyle{fancy}
\renewcommand{\sectionmark}[1]{\markright{\thesection}}
\rhead{Page \thepage\ of \pageref{LastPage}}
\lhead{}
\lfoot{CS 302 | Spring 2010}
\cfoot{}
\renewcommand{\footrulewidth}{0.4pt}

\tableofcontents

%\listoffigures
\newpage

\lhead{Joshua Gleason \& Josiah Humphrey}
\rhead{Page \thepage\ of \pageref{LastPage}}
\rfoot{Section\ \rightmark}
\cfoot{}
\lfoot{CS 302 | Spring 2010}
\renewcommand{\footrulewidth}{0.4pt}

\section{Introduction}

In this assignment, we looked at 3 different data structures that are very important for computer science. We examined
the usefulness and implementation of the heap, the priority queue, and the binary search tree. The program that we
designed to test these data structures were fairly trivial, but the challenge was implementing the assignment using the
data structures that we were supposed to implement. 

The first part of the assignment was to make a binary search tree that help users and their passwords. The binary search
tree was built using the name of the user as the node name. This allowed us to quickly search the list of users. This is
efficient when there are many users and when the program user needs to find a specific user quickly. The implementation
of the binary search tree we implemented was a template class that could handle any data structure. For this assignment,
we made our own data structure that included the name and the pass as string objects. This allowed us to do many useful
things such as compare. This was useful because it allowed the quick comparison of the string objects when building the
binary search tree. 

In the second part of the assignment, we were asked to implement an updated priority queue that allowed the users to
remove a specific item from the tree and to update the elements within the tree. This was done through inheritance. A
class called ``pqueue.h'' was defined that was a basic priority queue. The new updated queue, U\_PQType, was derived
from this class in order to include the two new functions. This derived class used the heap property of the priority
queue to implement the assignments challenge. We were asked to read in numbers from a file and place them into a heap.

\section{Use of Code}

In order to use this program, you will need data files that correspond to how we implemented the program. For assignment
one, there needs to be a data file that has a user name and password that are separated by white space. It should be in
the format of ``user1 pass1 \\n user2 pass2'' where \\n is an actual new line. We have provided a sample file for your
convenience. Once you load up the program, it asks you to enter a file name with the user info in it. It will do this
until you have correctly entered in a file name that is readable and in the correct format. Once this happens, you are
brought to a menu in which you can do the operations that are required. It should be fairly simple from here.

To use part 2 of the assignment, you will need to use a data file that has integers in it separated by white space. When
the program begins, it asks for a file name until it is given a suitable file name is given. Once inside the main menu,
you will have the options of what was assigned for this challenge. Once here, it should be fairly straight forward as to
what each menu item does.
\section{Functions}

%%%%%%%%%%%%%%%%%%%%%%%%%%%%%%%%%%%%%%%%
%%%%%%%       PART 1      %%%%%%%%%%%%%%
%%%%%%%%%%%%%%%%%%%%%%%%%%%%%%%%%%%%%%%%
\subsection{binaryTree.h}
\begin{description}

	\item{\textsc{binaryTree}}
\begin{lstlisting}
	binaryTree();
\end{lstlisting}
		\begin{description}

			\item{Purpose}

				Constructor, simply initializes root to NULL

			\item{Input}

				N/A

			\item{Output}

				N/A

			\item{Assumptions}

				N/A

		\end{description}
	\item{\textsc{~binaryTree}}
\begin{lstlisting}
	~binaryTree();
\end{lstlisting}
		\begin{description}

			\item{Purpose}

				Free all the memory in the tree

			\item{Input}

				N/A

			\item{Output}

				N/A

			\item{Assumptions}

				N/A

		\end{description}
	\item{\textsc{makeEmpty}}
\begin{lstlisting}
	void makeEmpty();
\end{lstlisting}
		\begin{description}

			\item{Purpose}
				
				Recursively remove all the nodes below the node.
			
			\item{Input}
			
				A node.

			\item{Output}

				N/A

			\item{Assumptions}

				Node is either NULL or points to an allocated node type.

		\end{description}
	\item{\textsc{isEmpty}}
\begin{lstlisting}
	bool isEmpty() const;
\end{lstlisting}
		\begin{description}

			\item{Purpose}

				Test if tree is empty.

			\item{Input}

				N/A

			\item{Output}

				Returns true if tree is empty, otherwise returns false.

			\item{Assumptions}

				N/A

		\end{description}
	\item{\textsc{numberOfNodes}}
\begin{lstlisting}
	int numberOfNodes() const;
\end{lstlisting}
		\begin{description}

			\item{Purpose}

				Count the number of nodes in the tree recursively.

			\item{Input}

				N/A

			\item{Output}

				Return the number of nodes in the tree.

			\item{Assumptions}

				N/A

		\end{description}
	\item{\textsc{retrieveItem}}
\begin{lstlisting}
	bool retrieveItem(iType&);
\end{lstlisting}
		\begin{description}

			\item{Purpose}

				Attempts to retrieve an item from the tree, return false if item is not found in the tree.

			\item{Input}

				An item that has the same "key" value as the item in question.

			\item{Output}

				Return true if item is in tree, false otherwise.

			\item{Assumptions}

				N/A

		\end{description}
	\item{\textsc{insertItem}}
\begin{lstlisting}
	void insertItem(iType);
\end{lstlisting}
		\begin{description}

			\item{Purpose}

				Create a new node and insert it into the tree with the value of the parameter.

			\item{Input}

				The value to be placed in the tree.

			\item{Output}

				N/A

			\item{Assumptions}

				N/A

		\end{description}
	\item{\textsc{deleteItem}}
\begin{lstlisting}
	void deleteItem(iType);
\end{lstlisting}
		\begin{description}

			\item{Purpose}

				Remove a node from the tree.

			\item{Input}

				An item with the same key value as the item to be removed.

			\item{Output}

				N/A

			\item{Assumptions}

				The value of the item must exist in the tree, use retrieveItem to test before calling this unless it is
				certain that the item is in the tree.

		\end{description}
	\item{\textsc{resetTree}}
\begin{lstlisting}
	void resetTree(oType);
\end{lstlisting}
		\begin{description}

			\item{Purpose}

				Generate a queue to traverse the tree in a particular order.

			\item{Input}

				An order type which can be IN\_ORDER, PRE\_ORDER, or POST\_ORDER, each one initializes a different
				queue.

			\item{Output}

				N/A

			\item{Assumptions}

				N/A

		\end{description}
	\item{\textsc{getNextItem}}
\begin{lstlisting}
	bool getNextItem(iType&, oType);
\end{lstlisting}
		\begin{description}

			\item{Purpose}

				Retrieve the next item in the queue corresponding to the value of oType.

			\item{Input}

				A reference parameter used to store the retrieved data and an order type of which queue to retrieve
				from.

			\item{Output}

				Return true if the item is the last item in the queue, also make the reference parameter equal to the
				value of the next item in the queue.

			\item{Assumptions}

				Assumes the proper queue as been initialized and is not empty.  Use the resetTree and isEmpty functions
				to ensure this is the case.

		\end{description}
	\item{\textsc{printTree}}
\begin{lstlisting}
	void printTree(ostream&) const;
\end{lstlisting}
		\begin{description}

			\item{Purpose}

				Prints the tree to the desired ostream type.

			\item{Input}

				An ostream object by reference, this could be cout or an ofstream.

			\item{Output}

				Outputs all values of the tree to the desired ostream.

			\item{Assumptions}

				Assumes that the left shift operator << has been overloaded for ostream to output the iType values
				properly.

		\end{description}
	\item{\textsc{countNodes}}
\begin{lstlisting}
	int countNodes(treeNode<iType>*);
\end{lstlisting}
		\begin{description}

			\item{Purpose}

				The client function called by numberOfNodes that does the recursive counting.

			\item{Input}

				A node pointer.

			\item{Output}

				Recursively determines the number of nodes connected to the parameter node plus 1.

			\item{Assumptions}

				Assumes parameter points to valid node or is NULL.

		\end{description}
	\item{\textsc{retrieve}}
\begin{lstlisting}
	bool retrieve(treeNode<iType>*, iType&);
\end{lstlisting}
		\begin{description}

			\item{Purpose}

				The recursive component to retrieveItem.

			\item{Input}

				A node pointer and the item in question.

			\item{Output}

				Returns true if item is found, false if it is not.

			\item{Assumptions}

				Assumes parameter points to valid node or is NULL.

		\end{description}
	\item{\textsc{insert}}
\begin{lstlisting}
	void insert(treeNode<iType>*&, iType);
\end{lstlisting}
		\begin{description}

			\item{Purpose}

				The recursive component to insertItem, recursively traverses the tree to find the appropriate place to
				put the node.

			\item{Input}

				A node pointer and the item to be inserted.
				
			\item{Output}

				N/A

			\item{Assumptions}

				Assumes the pointer points to a valid node or is NULL.

		\end{description}
	\item{\textsc{deleteOut}}
\begin{lstlisting}
	void deleteOut(treeNode<iType>*&, iType);
\end{lstlisting}
		\begin{description}

			\item{Purpose}

				This portion of the function finds the correct node to be deleted recursively, then calls deleteNode to
				actually do the delete.

			\item{Input}

				A node pointer and the item to be deleted.

			\item{Output}

				N/A

			\item{Assumptions}

				Assumes the value held by the iType parameter exists in the tree.

		\end{description}
	\item{\textsc{deleteNode}}
\begin{lstlisting}
	void deleteNode(treeNode<iType>*&);
\end{lstlisting}
		\begin{description}

			\item{Purpose}

				Delete the node without breaking any rules of the binary search tree.

			\item{Input}

				The node the be deleted.

			\item{Output}

				N/A

			\item{Assumptions}

				Assumes the node is part of a valid binary search tree.

		\end{description}
	\item{\textsc{getPredecessor}}
\begin{lstlisting}
	void getPredecessor(treeNode<iType>*, 
		Type&);
\end{lstlisting}
		\begin{description}

			\item{Purpose}

				Return the right most node of a left child.

			\item{Input}

				The left child of some node.

			\item{Output}

				Places the info from the right most child into the iType parameter.

			\item{Assumptions}

				Assumes the node is a valid node.

		\end{description}
	\item{\textsc{print}}
\begin{lstlisting}
	void print(treeNode<iType>*, ostream&);
\end{lstlisting}
		\begin{description}

			\item{Purpose}
				
				The recursive part of the print function, prints the tree inorder by printing the left child, the head,
				and then the right child.

			\item{Input}

				A node to be printed and an ostream object to print to.

			\item{Output}

				Prints all the values below the given node as well as the node using recursion.

			\item{Assumptions}
			
				Assumes the same as printTree as well as assuming the node is a valid node or NULL.

		\end{description}
	\item{\textsc{destroy}}
\begin{lstlisting}
	void destroy(treeNode<iType>*&);
\end{lstlisting}
		\begin{description}

			\item{Purpose}

				Recursively deallocate all values in the tree.

			\item{Input}

				A node pointer.

			\item{Output}

				N/A

			\item{Assumptions}

				Assumes the node pointer points to a valid node or is NULL.

		\end{description}
	\item{\textsc{copyTree}}
\begin{lstlisting}
	void copyTree(treeNode<iType>*&,
		treeNode<iType>*);
\end{lstlisting}
		\begin{description}

			\item{Purpose}

				Copy the values of the tree into another tree.

			\item{Input}

				The first parameter is the original tree, the second is the new tree.

			\item{Output}

				Copy all the nodes from one tree to the other.

			\item{Assumptions}

				Both trees are initialized and the left one is a valid tree.

		\end{description}
	\item{\textsc{preOrder}}
\begin{lstlisting}
	void preOrder(treeNode<iType>*&,
		queue<iType>&);
\end{lstlisting}
		\begin{description}

			\item{Purpose}
				
				Initialize the given queue with the values of the given tree traversed using pre-order traversal.

			\item{Input}

				A valid node pointer and an empty queue.

			\item{Output}

				Set the values of the queue to the pre-order traversal of the tree.

			\item{Assumptions}

				Assumes the queue is empty and the node pointer points to a valid node or is NULL.

		\end{description}
	\item{\textsc{inOrder}}
\begin{lstlisting}
	void inOrder(treeNode<iType>*&, 
		queue<iType>&);
\end{lstlisting}
		\begin{description}

		\item{Purpose}
			
			Initialize the given queue with the values of the given tree traversed using in-order traversal.

		\item{Input}

			A valid node pointer and an empty queue.

		\item{Output}

			Set the values of the queue to the in-order traversal of the tree.

		\item{Assumptions}

			Assumes the queue is empty and the node pointer points to a valid node or is NULL.

		\end{description}
	\item{\textsc{postOrder}}
\begin{lstlisting}
	void postOrder(treeNode<iType>*&, 
		queue<iType>&);
\end{lstlisting}
		\begin{description}

		\item{Purpose}
			
			Initialize the given queue with the values of the given tree traversed using post-order traversal.

		\item{Input}

			A valid node pointer and an empty queue.

		\item{Output}

			Set the values of the queue to the post-order traversal of the tree.

		\item{Assumptions}

			Assumes the queue is empty and the node pointer points to a valid node or is NULL.

		\end{description}

\end{description}
\subsection{user.h}
\begin{description}
	\item{\textsc{getName}}
\begin{lstlisting}
	string getName() const
\end{lstlisting}
		\begin{description}

			\item{Purpose}
			
				Return the name of the user as a string.

			\item{Input}

				N/A

			\item{Output}
			
				Returns a copy of the private name member.

			\item{Assumptions}

				N/A

		\end{description}
	\item{\textsc{getPass}}
\begin{lstlisting}
	string getPass() const
\end{lstlisting}
		\begin{description}

			\item{Purpose}

				Obtain a copy of the password string.

			\item{Input}

				N/A

			\item{Output}

				Returns a copy of the private pass member.

			\item{Assumptions}

				N/A

		\end{description}
	\item{\textsc{setName}}
\begin{lstlisting}
	void setName( string& rhs )
\end{lstlisting}
		\begin{description}

			\item{Purpose}

				Change the name of the user.

			\item{Input}

				A string which will replace the current name.

			\item{Output}

				N/A

			\item{Assumptions}

				N/A

		\end{description}
	\item{\textsc{setPass}}
\begin{lstlisting}
	void setPass( string& rhs )
\end{lstlisting}
		\begin{description}

			\item{Purpose}

				Change the password for the user.

			\item{Input}

				A string which will replace the current password.

			\item{Output}

				N/A

			\item{Assumptions}

				N/A

		\end{description}

	\item{\textsc{operator\(compare\)}}
\begin{lstlisting}
	bool operator>( const user& rhs )
	bool operator<( const user& rhs )
	bool operator>=( const user& rhs )
	bool operator<=( const user& rhs )
	bool operator==( const user& rhs )
\end{lstlisting}
		\begin{description}

			\item{Purpose}
				
				Used to compare one user to another.

			\item{Input}

				Using the string class's compare function return how the current user compares to the right hand side.

			\item{Output}

				Return correct bool value depending on which operator is being used.

			\item{Assumptions}

				N/A

		\end{description}

	\item{\textsc{operator<<}}
\begin{lstlisting}
	friend ostream& operator<<( ostream &out,
		user& rhs );
\end{lstlisting}
		\begin{description}

			\item{Purpose}

				The overloaded left shift operator for ostream using a user in the right hand side, this makes ostream
				object correctly output user information using the << operator.

			\item{Input}

				When used with an ostream object only the rhs is actually input.

			\item{Output}

				Print to the ostream the values inside of user.

			\item{Assumptions}

				N/A

		\end{description}

\end{description}
\subsection{part1.cpp}
\begin{description}

	\item{\textsc{readFile}}
\begin{lstlisting}
	bool readFile( string fileName,
		binaryTree<user>& tree );
\end{lstlisting}
		\begin{description}

			\item{Purpose}

				Read in a file and store it in the tree, this function randomizes the values to be inserted to prevent
				the tree from being unbalanced.

			\item{Input}

				The filename where the users are stored and a binary tree to store those users in.

			\item{Output}

				Store all the users in the file into the binary tree in random order, return true if file exists, false
				if it doesn't.

			\item{Assumptions}

				Assumes that if the file exists it does in fact contain a list of users and passwords separated only by
				whitespace.

		\end{description}
	\item{\textsc{storeTree}}
\begin{lstlisting}
	void storeTree( binaryTree<user>& tree,
		oType order, string fileName );
\end{lstlisting}
		\begin{description}

			\item{Purpose}

				Store the tree into a file in a particular order.

			\item{Input}

				A binary tree to be stores, an enumerated order
				$IN\_ORDER,$ 
				$PRE\_ORDER,$
				$POST\_ORDER$. Also a file name
				where to store the information.

			\item{Output}

				Output the values in the given order to a file.

			\item{Assumptions}

				N/A

		\end{description}
	\item{\textsc{promptForMenu}}
\begin{lstlisting}
	menuChoice promptForMenu();
\end{lstlisting}
		\begin{description}

			\item{Purpose}

				Prompt the user for a menu option and return the option when a valid choice is made.

			\item{Input}

				N/A

			\item{Output}

				Display the menu and continue to prompt the user until a valid value is entered.  Return the value as an
				enumerated type menuChoice which can be ADD, DEL, VERIFY, PRINT, IN\_ORD, PRE\_ORD, POST\_ORD, and EXIT.

			\item{Assumptions}

				N/A

		\end{description}
\end{description}
%%%%%%%%%%%%%%%%%%%%%%%%%%%%%%%%%%%%%%%%
%%%%%%%       PART 2      %%%%%%%%%%%%%%
%%%%%%%%%%%%%%%%%%%%%%%%%%%%%%%%%%%%%%%%
\subsection{heap.h}
\begin{description}

	\item{\textsc{reheapDown}}
\begin{lstlisting}
	void reheapDown(int root, int bottom);
\end{lstlisting}
		\begin{description}

			\item{Purpose}

				Assumes the root is messed up and fixes the heap property from the root down

			\item{Input}

				\begin{description}

					\item{root}
							
						The int that describes the root of the list

					\item{bottom}

						The int that describes the array value for the bottom

				\end{description}

			\item{Output}

				None

			\item{Assumptions}

				Assumes that the heap has been initialized and contains elements

		\end{description}
	\item{\textsc{reheapUp}}
\begin{lstlisting}
	void reheapUp(int root, int bottom);
\end{lstlisting}
		\begin{description}

			\item{Purpose}

				Fixes the heap property when something has been added at the bottom left of the tree. Fixes all the way
				to the top of the tree.

			\item{Input}

				\begin{description}

					\item{root}
							
						The int that describes the root of the list

					\item{bottom}

						The int that describes the array value for the bottom

				\end{description}
						
			\item{Output}

				None

			\item{Assumptions}
				
				Assumes that the heap has been initialized and contains elements

		\end{description}
	\item{\textsc{swap}}
\begin{lstlisting}
	void swap(ItemType &a, ItemType &b);
\end{lstlisting}
		\begin{description}

			\item{Purpose}

				Swaps two elements in the array

			\item{Input}

				\begin{description}

					\item{a}

						Item to be swapped

					\item{b}

						Item to be swapped

				\end{description}

			\item{Output}

				None

			\item{Assumptions}

				None

		\end{description}
\end{description}
\subsection{pqueue.h}
\begin{description}

	\item{\textsc{makeEmpty}}
\begin{lstlisting}
	void makeEmpty();
\end{lstlisting}
		\begin{description}

			\item{Purpose}

				Empties the entire heap.

			\item{Input}

				None

			\item{Output}

				None

			\item{Assumptions}

				None

		\end{description}
	\item{\textsc{isEmpty}}
\begin{lstlisting}
	bool isEmpty() const;
\end{lstlisting}
		\begin{description}

			\item{Purpose}

				Tells you if the heap is empty or not

			\item{Input}

				None

			\item{Output}

				Bool. True if empty, false if not empty

			\item{Assumptions}

				None

		\end{description}
	\item{\textsc{isFull}}
\begin{lstlisting}
	bool isFull() const;
\end{lstlisting}
		\begin{description}

			\item{Purpose}

				Tells you if the heap is full

			\item{Input}

				None

			\item{Output}

				Bool. True if full, false if not full.

			\item{Assumptions}

				None

		\end{description}
	\item{\textsc{enqueue}}
\begin{lstlisting}
	void enqueue(ItemType newItem);
\end{lstlisting}
		\begin{description}

			\item{Purpose}

				Puts an item into the heap

			\item{Input}

				\begin{description}

					\item{newItem}

						The item to be inserted into the heap

				\end{description}

			\item{Output}

				None

			\item{Assumptions}

				Assumes there is room in the heap

		\end{description}
	\item{\textsc{dequeue}}
\begin{lstlisting}
	void dequeue(ItemType& item);
\end{lstlisting}
		\begin{description}

			\item{Purpose}

				Pops out the root of the heap

			\item{Input}

				\begin{description}

					\item{item}

						The item that will hold the value that was located in the root of the heap

				\end{description}

			\item{Output}

				None

			\item{Assumptions}

				Assumes the heap is not empty

		\end{description}
\end{description}
\subsection{U\_PQType.h}
\begin{description}

	\item{\textsc{U\_PQType}}
\begin{lstlisting}
	U_PQType(int);
\end{lstlisting}
		\begin{description}

			\item{Purpose}

				Calls the constructor for PQType

			\item{Input}

				\begin{description}

					\item{max}

						The integer value to make for the new heap

				\end{description}

			\item{Output}

				None

			\item{Assumptions}

				None

		\end{description}
	\item{\textsc{Remove}}
\begin{lstlisting}
	void Remove(ItemType);
\end{lstlisting}
		\begin{description}

			\item{Purpose}

				Removes an item from a heap

			\item{Input}

				\begin{description}

					\item{item}

						The item that will be removed from the heap

				\end{description}

			\item{Output}

				None

			\item{Assumptions}

				None

		\end{description}
	\item{\textsc{Update}}
\begin{lstlisting}
	void Update(ItemType, ItemType);
\end{lstlisting}
		\begin{description}

			\item{Purpose}

				Updates the heap with a new value

			\item{Input}

				\begin{description}

					\item{item}

						The item that is to be replaced

					\item{newItem}

						The item that will replace the old item

				\end{description}

			\item{Output}

				None

			\item{Assumptions}

				None

		\end{description}
	\item{\textsc{printTree}}
\begin{lstlisting}
	void printTree(std::ostream&);
\end{lstlisting}
		\begin{description}

			\item{Purpose}

				Prints the tree to console

			\item{Input}

				\begin{description}

					\item{out}

						The ostream object that will hold the text that is written to the console

				\end{description}

			\item{Output}

				None

			\item{Assumptions}

				That there is a console to write to

		\end{description}
\end{description}
\subsection{part2.cpp}
\begin{description}
	\item{\textsc{readFile}}
\begin{lstlisting}
	bool readFile( string fileName, 
		U\_PQType<int>* &tree);
\end{lstlisting}
		\begin{description}

			\item{Purpose}

				Reads the file, counts how many numbers there are, initializes the heap and places the file contents
				into the heap

			\item{Input}

				\begin{description}

					\item{fileName}

						The string object that will hold the file name of the file to be opened

					\item{tree}

						A U\_PQTypw pointer that will be initialized inside the function

				\end{description}

			\item{Output}

				Bool telling you if the read was a success

			\item{Assumptions}

				None

		\end{description}
	\item{\textsc{promptForMenu}}
\begin{lstlisting}
	menuChoice promptForMenu();
\end{lstlisting}
		\begin{description}

			\item{Purpose}

				Displays the menu and sets up the case for the action menu

			\item{Input}

				None

			\item{Output}

				The menuChoice, which is an enumerated type

			\item{Assumptions}

				None

		\end{description}

\end{description}
\section{Bugs and Errors}

For the first section of the assignment there were no errors creating the binary tree, but there were however some
problems when I began trying to use the binary tree in the code.  The main problem was that the list of users we were
given was stored in order which caused some headaches when trying to create a binary tree because I was ending up with a
linked list rather than a tree.  The way I knew this was that the pre-order traversal was outputting the same values as
the in-order traversal.  To correct this issue I created a function that read the file into an array first and then;
using the property that any element in the array can be accessed in constant time; picked random elements in the array
and placed them into the tree which worked very nicely in fact.

Another rather silly issue I spent way too much time on was that I forgot that ostream is part of the std namespace. It
took nearly an hour of debugging to realize why I was receiving an entire page of cryptic compile errors.  After adding a
single line (using namespace std) to the beginning of the binaryTree.h file all those errors disappeared.

In part 2 of the assignment, the biggest bug that was found was that I could not figure out how to access the heaps
element pointer in the U\_PQType class. Even though the data members of pqueue were protected, the compiler would not
let me access them. This bug was avoided by using a different approach to the problem that worked just as well in the
end. 

Another problem that was encountered was the correct syntax for using a template class as a pointer and passing that
pointer by reference to a function. This was not so much an error, but more as an experience that helped me understand
the syntax more clearly. 

\section{What was Learned}

The main thing that was learned in this assignment was the implementation and use of different data structures. We were
exposed to the binary search tree implemented as a link list and a heap and priority queue implemented as a dynamic
array. This provided valuable experience to the students in the data structures involved and helped them understand what
makes a well designed data structure and when to use the appropriate data structure. 

The students also furthered their knowledge of templates and how to implement complex data structures as templates. This
knowledge will help them in the future because they will be equipped with the knowledge of how to implement complex data
structures using templates instead of rewriting data structures to handle every object type. The students are therefore
better equipped to overcome the challenges that they will meet in the real world and are better prospective employees as
a result of their knowledge and practical application of using the data structures involved in this programming
assignment.

\section{Division of Labor}

For this assignment, Josh did part 1 of the assignment. He wrote the binary tree implementation, the user class, and the
driver. The binary tree was implemented using a linked list approach to allow for it's ease of expandability. He also
worked on the documentation for those parts and helped overall with the report and with its writing.

Josiah wrote part 2 of the assignment. This involved writing the pqueue implementation which relied on the heap
implementation that he also wrote. Josiah also wrote the U\_PQType class that was derived from the pqueue class. Josiah
implemented all of this using dynamic arrays so that the heap could be implemented efficiently. Josiah also worked on
the report for the assignment and wrote the documentation for the functions that he implemented. 

\section{Inner workings of Remove and Update}

The way the Remove function was implemented for the U\_PQType is that it actually dequeues the entire list and then
enqueues only the values that are not equal to the value being removed.  While we do realize this is definitely not the
most efficient method to accomplish this, we were rather pressed for time due to many end of the semester projects,
tests, and assignments.  The time complexity for this function is
$\mathcal{O}\Big(n*(log_2(n)+log_2(n)\Big)=\mathcal{O}\Big(n*log_2(n)\Big)$ to dequeue the entire list then the same to
enqueue it again. This makes the entire function a total of...

$\mathcal{O}\Big(n*log_2(n)\Big)$

The method used for the Update function nearly identical to that of Remove where we simply dequeued the entire list and
enqueued everything back onto it.  The only difference is that when dequeueing the list, when we came across the desired
item to be updated we changed it's value.  The efficiency is therefore the exact same being...

$\mathcal{O}\Big(n*log_2(n)\Big)$

\end{document}
