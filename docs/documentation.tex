\documentclass[pdftex, 11pt]{article}

\usepackage[pdftex]{graphicx}
\usepackage{listings}
\usepackage{fancyhdr}
\usepackage{lastpage}
\usepackage{fullpage}
\usepackage{color}
\usepackage{amsmath}
\usepackage[pdftex,bookmarks=true,colorlinks=true,linkcolor=blue]{hyperref}
\usepackage{subfig}

%this sets the line at the header
\setlength{\headheight}{15.2pt}

\newcommand{\HRule}{\rule{\linewidth}{0.5mm}}

% This is all for formatting and making the Table of Contents according to 
% spec. Don't play with it.
\makeatletter
\renewcommand\l@section[2]{%
  \ifnum \c@tocdepth >\z@
    \addpenalty\@secpenalty
    \addvspace{1.0em \@plus\p@}%
    \setlength\@tempdima{1.5em}%
    \begingroup
      \parindent \z@ \rightskip \@pnumwidth
      \parfillskip -\@pnumwidth
      \leavevmode \bfseries
      \advance\leftskip\@tempdima
      \hskip -\leftskip
      #1\nobreak\ 
      \leaders\hbox{$\m@th\mkern \@dotsep mu\hbox{.}\mkern \@dotsep mu$}
     \hfil \nobreak\hb@xt@\@pnumwidth{\hss #2}\par
    \endgroup
  \fi}
\makeatother



\begin{document}

%import the title page
\begin{titlepage}
	\begin{center}

		% Upper part of the page
		\textsc{\LARGE University of Nevada, Reno}\\[.5cm]
		\includegraphics[width=0.15\textwidth]{./logo.png}\\[.5cm]

		\textsc{\large CS 302 | Data Structures } \\[.5cm]

		% Title
		\HRule \\[0.4cm]
		{ \huge \bfseries Assignment \#5}\\[0.4cm]

		\HRule \\[1.5cm]

		% Author and supervisor
		\begin{minipage}{0.4\textwidth}
			\begin{flushleft} \large
				\emph{Students:}\\
				Joshua \textsc{Gleason}\\
				Josiah \textsc{Humphrey}
			\end{flushleft}
		\end{minipage}
		\begin{minipage}{0.4\textwidth}
			\begin{flushright} \large
				\emph{Instructor:} \\
				Dr. George \textsc{Bebis}
			\end{flushright}
		\end{minipage}

		\vfill

		% Bottom of the page
		{\large \today}

	\end{center}

\end{titlepage}


%headers, footers, and table of contents
\pagestyle{fancy}
\renewcommand{\sectionmark}[1]{\markright{\thesection}}
\rhead{Page \thepage\ of \pageref{LastPage}}
\lhead{}
\lfoot{CS 302 | Spring 2010}
\cfoot{}
\renewcommand{\footrulewidth}{0.4pt}

\tableofcontents

%\listoffigures
\newpage

\lhead{Joshua Gleason \& Josiah Humphrey}
\rhead{Page \thepage\ of \pageref{LastPage}}
\rfoot{Section\ \rightmark}
\cfoot{}
\lfoot{CS 302 | Spring 2010}
\renewcommand{\footrulewidth}{0.4pt}

\section{Introduction}

In this programming assignment, we continued and built upon image processing. We took out previous programming
assignment and added more functionality. Our goals were to represent the regions found by connected components in a list
and to compute a number of useful properties to characterize their position, orientation, shape, and intensity. This was
a challenge that tested our ability to understand the data structures and image processing. We were also asked to use our
skills to manipulate items in lists, use templates, and learn about feature extraction and classification. 

In this assignment, we learned about geometric properties of objects and how to objectively classify them using
mathematical formulas. We were able to implement the equations that found various geometric properties of objects found
in images. Since this is a introduction to image processing, we were only asked to implement geometric and intensity
properties only. Using more advanced techniques we could have done more complicated things.

To calculate these properties, we used a technique from probability theory called moments. A moment is defined as 
\[
M_{p,q} = \sum_{i,j \in R} i^{p} j^{q}
\]

This was the basis for many of our calculations and was implemented as a generic function that would take any numbers
for $p$ and $q$.

\section{Use of Code}

The use of this code should be fairly straight forward. The main menu is easy to understand, just select the option you
want and hit enter. To load images into the programs image registers, you can either do it with the menu option, or on
the command line. To load images on the command line, just enter them as arguments. For instance, you could do \texttt{\$
./main.out images/hubble1.pgm}. You can enter as many images as you want on the command line like so: \texttt{\$
./main.out images/hubble1.pgm images/hubble2.pgm images/hubble3.pgm}

To classify the regions in an image, you need to choose the classify region option in the menu. Once you do this, you
will be able to choose which of the classifiers you would like to do. You can do multiple things to the same image. Once
you are done classifying the image, make sure you choose the save image option that is in that sub-menu to save all of
the changes. You will then need to save the image at the top of the main menu in order to output the image. 

One note about the classifiers. There are minimum and maximum values for each function, but to make them easier to use,
if the user inputs bounds that are below the minimum and/or above the maximum, it defaults to the minimum and/or maximum
value respectively. This was done to create a more user friendly program that is capable of accepting data that the
user intends. For instance, if the user wants to include the maximum size, you can just enter "99999999" and it will
interpret that input as a maximum value, but the program will automatically make that input sane for the environment.

\section{Functions}

\subsection{Image.h}

\begin{description}

	\item{\textsc{regionType()}}
		\begin{description}
\begin{lstlisting}
	RegionType();
\end{lstlisting}

			\item{Purpose}

				Sets all of the data to 0 or 0.0.

			\item{Input}

				None

			\item{Output}

				None

			\item{Assumption}

				None

		\end{description}

		RegionType(const RegionType<pType>\&);
	\item{\textsc{operator>}}
		\begin{description}
\begin{lstlisting}
	bool operator>(const RegionType<pType> &rhs) const;
\end{lstlisting}

			\item{Purpose}

				Overloaded function for greater than.

			\item{Input}

				A RegionType object to compare to

			\item{Output}

				Bool value based on the output of the comparison

			\item{Assumption}

				None

		\end{description}

	\item{\textsc{operator<}}
		\begin{description}
\begin{lstlisting}
	bool operator<(const RegionType<pType> &rhs) const;
\end{lstlisting}

			\item{Purpose}

				Overloaded function for less than.

			\item{Input}

				A RegionType object to compare to

			\item{Output}

				Bool value based on the output of the comparison

			\item{Assumption}

				None

		\end{description}

	\item{\textsc{operator>=}}
		\begin{description}
\begin{lstlisting}
	bool operator>=(const RegionType<pType> &rhs) const;
\end{lstlisting}

			\item{Purpose}

				Overloaded function for greater than or equal

			\item{Input}

				A RegionType object to compare to

			\item{Output}

				Bool value based on the output of the comparison

			\item{Assumption}

				None

		\end{description}

	\item{\textsc{operator<=}}
		\begin{description}
\begin{lstlisting}
	bool operator<=(const RegionType<pType> &rhs) const;
\end{lstlisting}

			\item{Purpose}

				Overloaded function for less than or equal

			\item{Input}

				A RegionType object to compare to

			\item{Output}

				Bool value based on the output of the comparison

			\item{Assumption}

				None

		\end{description}

	\item{\textsc{operator==}}
		\begin{description}
\begin{lstlisting}
	bool operator==(const RegionType<pType> &rhs) const;
\end{lstlisting}

			\item{Purpose}

				Overloaded function for equal to

			\item{Input}

				A RegionType object to compare to

			\item{Output}

				Bool value based on the output of the comparison

			\item{Assumption}

				None

		\end{description}

	\item{\textsc{operator= }}
		\begin{description}
\begin{lstlisting}
	RegionTypep& operator=(const RegionType<pType> &rhs);
\end{lstlisting}

			\item{Purpose}

				The overloaded $=$ sign that will copy the object on the left into the object on the right

			\item{Input}

				The object to be copied from

			\item{Output}

				The object itself to allow chaining

			\item{Assumption}

				None

		\end{description}

	\item{\textsc{setData}}
		\begin{description}
\begin{lstlisting}
	void setData( const ImageType<pType>& );
\end{lstlisting}

			\item{Purpose}

				The function that calls all of the other functions to set all of the data members

			\item{Input}

				An image of some sort

			\item{Output}

				None

			\item{Assumption}

				Assumes the picture is valid, but it will work for an image as long as the image exists.

		\end{description}

	\item{\textsc{getCentroidR}}
		\begin{description}
\begin{lstlisting}
	double getCentroidR() const;
\end{lstlisting}

			\item{Purpose}

				Gets the $R$ centroid for the image

			\item{Input}

				None

			\item{Output}

				The double value for the centroid

			\item{Assumption}

				None

		\end{description}

	\item{\textsc{getCentroidC}}
		\begin{description}
\begin{lstlisting}
	double getCentroidC() const;
\end{lstlisting}

			\item{Purpose}

				Gets the $C$ centroid for the image

			\item{Input}

				None

			\item{Output}

				None

			\item{Assumption}

				None

		\end{description}

	\item{\textsc{getSize}}
		\begin{description}
\begin{lstlisting}
	int getSize() const;
\end{lstlisting}

			\item{Purpose}

				Gets the size of the region using the moment calculation

			\item{Input}

				None

			\item{Output}

				The integer value for the size

			\item{Assumption}

				None

		\end{description}

	\item{\textsc{getOrientation}}
		\begin{description}
\begin{lstlisting}
	double getOrientation() const;
\end{lstlisting}

			\item{Purpose}

				Gets the orientation for a region

			\item{Input}

				None

			\item{Output}

				The double value for the orientation of the region

			\item{Assumption}

				None

		\end{description}

	\item{\textsc{getEccentricity}}
		\begin{description}
\begin{lstlisting}
	double getEccentricity() const;
\end{lstlisting}

			\item{Purpose}

				Gets the eccentricity value for the region

			\item{Input}

				None

			\item{Output}

				The double value for the eccentricity

			\item{Assumption}

				None

		\end{description}

	\item{\textsc{getMeanVal}}
		\begin{description}
\begin{lstlisting}
	pType getMeanVal() const;
\end{lstlisting}

			\item{Purpose}

				Gets the mean pixel value

			\item{Input}

				None

			\item{Output}

				The average pixel value returned as a pixelType



			\item{Assumption}
				None

		\end{description}

	\item{\textsc{getMinVal}}
		\begin{description}
\begin{lstlisting}
	pType getMinVal() const;
\end{lstlisting}

			\item{Purpose}

				Gets the minimum value for the region

			\item{Input}

				None

			\item{Output}

				The pixelType value for the minimum pixel value

			\item{Assumption}

				None

		\end{description}

	\item{\textsc{getMaxVal}}
		\begin{description}
\begin{lstlisting}
	pType getMaxVal() const;
\end{lstlisting}

			\item{Purpose}

				Gets the maximum value for the region

			\item{Input}

				None

			\item{Output}

				The pixelType value for the maximum pixel value

			\item{Assumption}

				None

		\end{description}

	\item{\textsc{moment}}
		\begin{description}
\begin{lstlisting}
	double moment(int, int);
\end{lstlisting}

			\item{Purpose}

				Calculates the moment using this formula:
\[
M_{p,q} = \sum_{i,j \in R} i^{p} j^{q}
\]

			\item{Input}

				Two ints that correspond to $p$ and $q$ in the equation

			\item{Output}

				The double value that is calculated by the function.				

			\item{Assumption}

				A natural number for $p$ and $q$.

		\end{description}

	\item{\textsc{mu}}
		\begin{description}
\begin{lstlisting}
	double mu(int, int);
\end{lstlisting}

			\item{Purpose}

				Calculates the equation defined by
				\[
				\mu_{p,q} = \sum_{i,r \in R} (i-\bar{x})^p (j-\bar{y})^q
				\]

			\item{Input}
				
				Two ints for $p$ and $q$

			\item{Output}

				The double value calculated by $\mu$

			\item{Assumption}
				Natural numbers for the ints

		\end{description}

	\item{\textsc{xyBar}}
		\begin{description}
\begin{lstlisting}
	void xyBar();
\end{lstlisting}

			\item{Purpose}

				Calculates the equation defined by 
				\[
				\bar{x} = \frac{M_{1,0}}{M_{0,0}}
				\]
				and 
				\[
				\bar{y} = \frac{M_{0,1}}{M_{0,0}}
				\]

			\item{Input}
			
				None

			\item{Output}

				None

			\item{Assumption}

				None

		\end{description}

	\item{\textsc{lambda}}
		\begin{description}
\begin{lstlisting}
	void lambda();
\end{lstlisting}

			\item{Purpose}

				Calculates the equation defined by
				\[
				\lambda_{max} = \frac{1}{2} (\mu_{2,0} + \mu_{0,2}) + \frac{1}{2} \sqrt{ \mu_{2,0}^2 \mu_{0,2}^2 -
				2\mu_{0,2} \mu_{2,0} + 4 \mu_{1,1}^2 }
				\]
				and
				\[
				\lambda_{min} = \frac{1}{2} (\mu_{2,0} + \mu_{0,2}) - \frac{1}{2} \sqrt{ \mu_{2,0}^2 \mu_{0,2}^2 -
				2\mu_{0,2} \mu_{2,0} + 4 \mu_{1,1}^2 }
				\]

			\item{Input}

				None

			\item{Output}

				None

			\item{Assumption}

				None

		\end{description}

	\item{\textsc{theta}}
		\begin{description}
\begin{lstlisting}
	void theta();
\end{lstlisting}

			\item{Purpose}

				Calculates the equation defined by
				\[
				\theta = \tan^{-1} { \frac{\lambda_{max} - \mu_{2,0}}{\mu_{1,1}} }	
				\]

			\item{Input}

				None

			\item{Output}

				None

			\item{Assumption}

				None

		\end{description}

	\item{\textsc{epsilon}}
		\begin{description}
\begin{lstlisting}
	void epsilon();
\end{lstlisting}

			\item{Purpose}

				Calculates the eccentricity defined by the equation
				\[
				\varepsilon = \sqrt{\frac{\lambda_{max}}{\lambda_{min}}}
				\]

			\item{Input}

			\item{Output}

			\item{Assumption}

		\end{description}

\end{description}

\subsection{sortedList.h}

\begin{description}

	\item{\textsc{sortedList()}}
\begin{lstlisting}
	sortedList();
\end{lstlisting}

		\begin{description}

			\item{Purpose}

				Sets the data to null and length to 0

			\item{Input}

				None

			\item{Output}

				None

			\item{Assumption}

				None

		\end{description}

	\item{\textsc{~sortedList}}
\begin{lstlisting}
	~sortedList();
\end{lstlisting}

		\begin{description}

			\item{Purpose}

				Deletes the list

			\item{Input}

				None

			\item{Output}

				None

			\item{Assumption}

				None

		\end{description}

	\item{\textsc{getLength}}
\begin{lstlisting}
	int getLength();
\end{lstlisting}

		\begin{description}

			\item{Purpose}

				Returns the length of the list

			\item{Input}

				None

			\item{Output}

				Returns the length of the list

			\item{Assumption}

				None

		\end{description}

	\item{\textsc{makeEmpty}}
\begin{lstlisting}
	void makeEmpty();
\end{lstlisting}

		\begin{description}

			\item{Purpose}

				Empties the list

			\item{Input}

				None

			\item{Output}

				None
			\item{Assumption}

				None

		\end{description}

	\item{\textsc{retrieveItem}}
\begin{lstlisting}
	bool retrieveItem( T& );
\end{lstlisting}

		\begin{description}

			\item{Purpose}

				Checks to see if the item passed to the function is in the list

			\item{Input}

				The item to be checked if it exists in the list

			\item{Output}

				The bool to say if the item was found

			\item{Assumption}

				None

		\end{description}

	\item{\textsc{insertItem}}
\begin{lstlisting}
	void insertItem( T );
\end{lstlisting}

		\begin{description}

			\item{Purpose}

				Inserts the item into the correct place into the list

			\item{Input}

				The item to be inserted

			\item{Output}

				None

			\item{Assumption}

				None

		\end{description}

	\item{\textsc{deleteItem}}
\begin{lstlisting}
	void deleteItem( T );
\end{lstlisting}

		\begin{description}

			\item{Purpose}

				Deletes the item and relinks the list to preserve the sorted attribute

			\item{Input}

				The item to be deleted

			\item{Output}

				None

			\item{Assumption}

				None

		\end{description}

	\item{\textsc{reset}}
\begin{lstlisting}
	void reset(');
\end{lstlisting}

		\begin{description}

			\item{Purpose}

				Resets the head pointer to be at the top of the list. This needs to be done before running through the
				list.

			\item{Input}

				None

			\item{Output}

				None

			\item{Assumption}

				None

		\end{description}

	\item{\textsc{isEmpty}}
\begin{lstlisting}
	bool isEmpty();
\end{lstlisting}

		\begin{description}

			\item{Purpose}

				A function to tell you if the list is empty

			\item{Input}

				None

			\item{Output}

				Bool telling you if the list is empty

			\item{Assumption}

				None

		\end{description}

	\item{\textsc{atEnd}}
\begin{lstlisting}
	bool atEnd();
\end{lstlisting}

		\begin{description}

			\item{Purpose}

				Tells the user if the current list pointer is pointing at the last element. Useful for using a while
				loop to do things to the list.

			\item{Input}

				None

			\item{Output}

				Bool telling you if the current list item is pointing to NULL

			\item{Assumption}

				None

		\end{description}

	\item{\textsc{getNextItem}}
\begin{lstlisting}
	T getNextItem();
\end{lstlisting}

		\begin{description}

			\item{Purpose}

				Gets the next item in the list

			\item{Input}

				None

			\item{Output}

				The item that comes next in the list

			\item{Assumption}

				None

		\end{description}


	\item{\textsc{operator=}}
\begin{lstlisting}
	sortedList<T>& operator=(const sortedList<T>&);
\end{lstlisting}

		\begin{description}

			\item{Purpose}

				Copies the list into another list

			\item{Input}

				The list to be copied from

			\item{Output}

				The object to allow for chaining

			\item{Assumption}

				None

		\end{description}

\end{description}

\subsection{list.h}

\begin{description}

	\item{\textsc{list()}}
\begin{lstlisting}
	sortedList();
\end{lstlisting}

		\begin{description}

			\item{Purpose}

				Sets the data to null and length to 0

			\item{Input}

				None

			\item{Output}

				None

			\item{Assumption}

				None

		\end{description}

	\item{\textsc{~ist}}
\begin{lstlisting}
	~sortedList() { makeEmpty(); }
\end{lstlisting}

		\begin{description}

			\item{Purpose}

				Deletes the list

			\item{Input}

				None

			\item{Output}

				None

			\item{Assumption}

				None

		\end{description}

	\item{\textsc{getLength}}
\begin{lstlisting}
	int getLength();
\end{lstlisting}

		\begin{description}

			\item{Purpose}

				Returns the length of the list

			\item{Input}

				None

			\item{Output}

				Returns the length of the list

			\item{Assumption}

				None

		\end{description}

	\item{\textsc{makeEmpty}}
\begin{lstlisting}
	void makeEmpty();
\end{lstlisting}

		\begin{description}

			\item{Purpose}

				Empties the list

			\item{Input}

				None

			\item{Output}

				None
			\item{Assumption}

				None

		\end{description}

	\item{\textsc{retrieveItem}}
\begin{lstlisting}
	bool retrieveItem( T& );
\end{lstlisting}

		\begin{description}

			\item{Purpose}

				Checks to see if the item passed to the function is in the list

			\item{Input}

				The item to be checked if it exists in the list

			\item{Output}

				The bool to say if the item was found

			\item{Assumption}

				None

		\end{description}

	\item{\textsc{insertItem}}
\begin{lstlisting}
	void insertItem( T );
\end{lstlisting}

		\begin{description}

			\item{Purpose}

				Inserts the item into the list

			\item{Input}

				The item to be inserted

			\item{Output}

				None

			\item{Assumption}

				None

		\end{description}

	\item{\textsc{deleteItem}}
\begin{lstlisting}
	void deleteItem( T );
\end{lstlisting}

		\begin{description}

			\item{Purpose}

				Deletes the item and relinks the list 

			\item{Input}

				The item to be deleted

			\item{Output}

				None

			\item{Assumption}

				None

		\end{description}

	\item{\textsc{reset}}
\begin{lstlisting}
	void reset();
\end{lstlisting}

		\begin{description}

			\item{Purpose}

				Resets the head pointer to be at the top of the list. This needs to be done before running through the
				list.

			\item{Input}

				None

			\item{Output}

				None

			\item{Assumption}

				None

		\end{description}

	\item{\textsc{isEmpty}}
\begin{lstlisting}
	bool isEmpty();
\end{lstlisting}

		\begin{description}

			\item{Purpose}

				A function to tell you if the list is empty

			\item{Input}

				None

			\item{Output}

				Bool telling you if the list is empty

			\item{Assumption}

				None

		\end{description}

	\item{\textsc{atEnd}}
\begin{lstlisting}
	bool atEnd();
\end{lstlisting}

		\begin{description}

			\item{Purpose}

				Tells the user if the current list pointer is pointing at the last element. Useful for using a while
				loop to do things to the list.

			\item{Input}

				None

			\item{Output}

				Bool telling you if the current list item is pointing to NULL

			\item{Assumption}

				None

		\end{description}

	\item{\textsc{getNextItem}}
\begin{lstlisting}
	T getNextItem();
\end{lstlisting}

		\begin{description}

			\item{Purpose}

				Gets the next item in the list

			\item{Input}

				None

			\item{Output}

				The item that comes next in the list

			\item{Assumption}

				None

		\end{description}


\item{\textsc{operator=}}
\begin{lstlisting}
	sortedList<T>& operator=(const sortedList<T>&);
		\end{lstlisting}

		\begin{description}

			\item{Purpose}

				Copies the list into another list

			\item{Input}

				The list to be copied from

			\item{Output}

				The object to allow for chaining

			\item{Assumption}

				None

		\end{description}

\end{description}

\subsection{driver.cpp}

\begin{description}

	\item{\textsc{computeComponents}}
\begin{lstlisting}
	int computeComponents( ImageType<pType>,
		sortedList<RegionType<pType> >& );
\end{lstlisting}
		\begin{description}

			\item{Purpose}

				The main calling function that will get all of the regions and classify them

			\item{Input}

				A single image and a list of regions

			\item{Output}

				Fills the refion list with all the regions in the image and returns the total number of regions

			\item{Assumption}

				That the image is a valid image and sorted list is initialized

		\end{description}
		
	\item{\textsc{findComponentsDFS}}
\begin{lstlisting}
	void findComponentsDFS( ImageType<pType>, 
		ImageType<pType>&, int, int, pType,
		RegionType<pType>&, const ImageType<pType>& );
\end{lstlisting}
		\begin{description}

			\item{Purpose}

				Computes the region's attributes and stores them in the region node

			\item{Input}

				Input image, output image, location of a pixel in the region, a region node to store the data in, the
				original image

			\item{Output}

				Fills the region object as well as flooding the region in the output image

			\item{Assumption}

				That the input is already thresholded and the location is part of the region.

		\end{description}

	\item{\textsc{deleteSmallRegions}}
\begin{lstlisting}
	void deleteSmallRegions( sortedList<RegionType<pType> >&,
		int );
\end{lstlisting}

		\begin{description}

			\item{Purpose}

				Finds the regions in an image that are below in size of a certain threshold and deletes them

			\item{Input}

				A list of regions 

			\item{Output}

				Prints a summary of all the regions to the screen

			\item{Assumption}

				That the region list is a valid list

		\end{description}

	\item{\textsc{printSummary}}
\begin{lstlisting}
	void printSummary( sortedList<RegionType<pType> >& );
\end{lstlisting}

		\begin{description}

			\item{Purpose}

				Prints a summary of the regions to the screen

			\item{Input}

				A list of regions

			\item{Output}

				A summary of all of the regions

			\item{Assumption}

				That the list of regions is valid

		\end{description}


\end{description}


\section{Bugs and Errors}

During the creating of this program, there was one single bug that took a very, very long time to track down, following
is a detailed explanation of the bug and the methods used to track down and repair it.

The problem originally manifested itself as a segmentation fault when the choice to 'Classify Regions' was selected in
the main menu.  At first I looked through the classifyRegions function for any obvious problems, after that search came
up empty I began using the GDB debugger to track down the fatal error.

The first thing I needed to know was where the actual error was occurring, so I executed the program in GDB.  After the
re-creating the segmentation fault I found that the crash was occurring a conditional statement inside of the $==$
operator overload function inside of the Region Type class.  By examining parameters passed to the function I discovered
that the right hand side was actually an invalid value, printing the address of the parameter I found the value was
actually NULL.  This seemed very strange, so I used GDB's backtrace command, which indicated that the comparison was
taking place in the deleteItem function of the sortedList class or more specific the list of regions for the image.

Before debugging further, I pondered the recently acquired information and came to a hypothesis.  I believed that the
deleteItem function was not finding the value that it was passed even thought the RegionType values were being directly
taken from the list of regions.  This was the only way I could conceive the $==$ operator being passed NULL from
deleteItem.  Some more debugging was definitely needed to verify this claim and also answer some other questions if this
was the case.

After setting a breakpoint in the deleteItem function I ran the program and selected the Classify Regions option.  The
program paused at the first breakpoint where I obtained some very interesting information about the RegionType in
question.  I ran the command \texttt{print *this} in GDB so that I could quickly see all of the private members of the
current object.  To my surprise one of the values was definitely invalid, which may explain why the $==$ operator never
returned true, even if the values had the same data members.  What would happen if you tried to compare two invalid double
values, even if they were copies of one another?  I had to determine the answer to this question.  By continuing execution
I found where two regions had all the same valid data members and when finished the $==$ function I discovered that the
returned value was false, which would explain why deleteItem never found the right value.

At this point I was feeling pretty good about having narrowed down the problem to a calculation issue, but why was I
getting invalid values for eccentricity for some of the regions?  To determine this I set a breakpoint in the
setData function of RegionType and recreated the error yet again.  To my surprise the first region had some invalid
values, but I also noticed that the value for lambdaMin was zero; I thought I recalled the eccentricity requiring
dividing by lambdaMin, so I checked it out.  I verified that this was the indeed true, so I decided to find why
lambdaMin was being set to zero.  After using similar techniques I discovered that lambdaMin was zero because the central
moment was returning zero for regions of width or height one.  I determined that the calculation was correct because I
had received the same value when I did it by hand, so the problem wasn't actually a coding problem, it was a problem with
the function equation.  To fix this problem, I added an exception to prevent dividing by zero by adding one to the
numerator and denominator if the central moment of 1,1 returned zero.  This single bug took nearly two hours to track
down, but after discovering the cause I at least feel much more compentent with GDB.


\section{What was Learned}

In this lab, the students learned about classifying regions in images. The students also learned about moments from
probability theory. The students combined these aspects of problem solving to come up with a solution to make
probabilistic calculations on regions found in an image.  The students used these tools to better understand the
application and development of image processing. The students were able to combine these tools successfully to implement
a working program that can correctly classify regions found within an image. To store the regions unique data, the
students implemented a sorted list and an unsorted list that contained the $x,y$ coordinates of the regions and a sorted
list of the regions and their associated data. This helped the students better understand data structures and helped the
students to know how to implement and develop a data structure. Since it was suggested to template these data
structures, the students choose to do this and had a template for the unsorted and sorted list types. This proved to
further the student's knowledge and prowess of templates in C++.  The students also learned how to embed other objects
into objects. This was done with the region object that had embedded in it an unsorted list to hold the region pixel
locations. This was not too hard, but still taught the students how to embed objects in objects.

\section{Division of Labor}

For this assignment, the labor was divided equally among the partners. Each student contributed equally to the
production and development of the code. The work was divided after the assignment was announced, and the students each
had about the same work to be done.

Joshua was responsible for writing the sorted list and Josiah wrote the unsorted list.  For the calculations Josiah
wrote most of the functions in RegionType dealing with moment and central moments and Joshua wrote the intensity
calculations.  Joshua also wrote most of the extra curses based code in driver including printSummary, both students
contributed equally to the classifyRegions function.  The documentation was split evenly while both students put their
own part into most of the sections and reviewed all of the sections.

\section{Extra Credit}
\begin{description}
	\item{Sorted List}

		The implementation for the sorted list was completed and done using templates. The lists were some of the first
		things that were templated because they were needed in order to complete the rest of the assignment. The sorted
		list uses a link list implementation and keeps the list sorted. While a array based implementation would have
		been faster to search and retrieve items, we choose a linked list implementation because we would not know how
		large the list would need to be. Therefore the only possible implementation that is efficient is a linked list
		implementation of the sorted list.

	\item{Unsorted List}

		The implementation for the unsorted list was completed and done using templates. The lists were some of the
		first things that were templated because they were needed in order to complete the rest of the assignment. The
		unsorted list uses a link list implementation and inserts at the fastest possible place because it is unsorted
		and it does not matter where the nodes are inserted. While a array based implementation would have been faster
		to search and retrieve items, we choose a linked list implementation because we would not know how large the
		list would need to be. Therefore the only possible implementation that is efficient is a linked list
		implementation of the unsorted list.

\end{description}

\end{document}
