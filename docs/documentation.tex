\documentclass[pdftex, 11pt]{article}

\usepackage[pdftex]{graphicx}
\usepackage{listings}
\usepackage{fancyhdr}
\usepackage{lastpage}
\usepackage{fullpage}
\usepackage{color}
\usepackage[pdftex,bookmarks=true,colorlinks=true,linkcolor=blue]{hyperref}

%this sets the line at the header
\setlength{\headheight}{15.2pt}

\newcommand{\HRule}{\rule{\linewidth}{0.5mm}}

% This is all for formatting and making the Table of Contents according to 
% spec. Don't play with it.
\makeatletter
\renewcommand\l@section[2]{%
  \ifnum \c@tocdepth >\z@
    \addpenalty\@secpenalty
    \addvspace{1.0em \@plus\p@}%
    \setlength\@tempdima{1.5em}%
    \begingroup
      \parindent \z@ \rightskip \@pnumwidth
      \parfillskip -\@pnumwidth
      \leavevmode \bfseries
      \advance\leftskip\@tempdima
      \hskip -\leftskip
      #1\nobreak\ 
      \leaders\hbox{$\m@th\mkern \@dotsep mu\hbox{.}\mkern \@dotsep mu$}
     \hfil \nobreak\hb@xt@\@pnumwidth{\hss #2}\par
    \endgroup
  \fi}
\makeatother



\begin{document}
%import the title page
\begin{titlepage}
	\begin{center}

		% Upper part of the page
		\textsc{\LARGE University of Nevada, Reno}\\[.5cm]
		\includegraphics[width=0.15\textwidth]{./logo.png}\\[.5cm]

		\textsc{\large CS 302 | Data Structures } \\[.5cm]

		% Title
		\HRule \\[0.4cm]
		{ \huge \bfseries Assignment \#5}\\[0.4cm]

		\HRule \\[1.5cm]

		% Author and supervisor
		\begin{minipage}{0.4\textwidth}
			\begin{flushleft} \large
				\emph{Students:}\\
				Joshua \textsc{Gleason}\\
				Josiah \textsc{Humphrey}
			\end{flushleft}
		\end{minipage}
		\begin{minipage}{0.4\textwidth}
			\begin{flushright} \large
				\emph{Instructor:} \\
				Dr. George \textsc{Bebis}
			\end{flushright}
		\end{minipage}

		\vfill

		% Bottom of the page
		{\large \today}

	\end{center}

\end{titlepage}


%headers, footers, and table of contents
\pagestyle{fancy}
\renewcommand{\sectionmark}[1]{\markright{\thesection}}
\rhead{Page \thepage\ of \pageref{LastPage}}
\lhead{}
\lfoot{CS 302 | Spring 2010}
\cfoot{}
\renewcommand{\footrulewidth}{0.4pt}

\tableofcontents
\newpage

\lhead{Joshua Gleason \& Josiah Humphrey}
\rhead{Page \thepage\ of \pageref{LastPage}}
\rfoot{Section\ \rightmark}
%\rfoot{Section\ \if\relax\thesubsection\relax \rightmark \else \thesection\fi}
\cfoot{}
\lfoot{CS 302 | Spring 2010}
\renewcommand{\footrulewidth}{0.4pt}

\section{Introduction}


\section{Use of Code}


\section{Functions}

\subsection{Image.h}
\begin{description}

	\item{\textsc{constructor}}
		\begin{description}
			\item{Purpose}

				default constructor allocates no memory and sets the size to zero 

			\item{Input}

				None

			\item{Output}

				None

			\item{Assumptions}

				Sets everything to zero and 
				sets the pixelValue array to NULL

		\end{description}


	\item{\textsc{constructor with parameters}}
		\begin{description}
			\item{Purpose}

				change the dimenstions of the image, delete,
				and re-allocate memory if required

			\item{Input}

				An N, M, and Q value to set the new image to

			\item{Output}

				None

			\item{Assumptions}

				Sets the image to a certain size and intializes the
				image as a grid

		\end{description}



	\item{\textsc{desctructor}}
		\begin{description}
			\item{Purpose}

				Deletes and memory that has been dynamically allocated

			\item{Input}

				None

			\item{Output}

				None

			\item{Assumptions}

				Checks to see if the pixelValue array has been set
				if so, deletes

		\end{description}


	\item{\textsc{copy\_constructor}}
		\begin{description}
			\item{Purpose}
		
				Creates a new array absed on the thing to be copied
				then sets the pixelValue of the new object the same as
				the old image

			\item{Input}

				ImageType rhs is the old image to be copied over into
				the new array


			\item{Output}

				None

			\item{Assumptions}

				The old image must be passed as reference to prevent
				an infinate loop


		\end{description}


	\item{\textsc{operator=}}
		\begin{description}
			\item{Purpose}

				equal operator overload, this is basically
				the same as the copy constructor
				except it will likely have to 
				de-allocate memory before copying values, all
				this is decided in setImageInfo however

			\item{Input}

				imageType rhs which is the old iamge to be 
				copied over to the new image

			\item{Output}

				Returns the imageType obejct so that
				equal chaining can be implemented


			\item{Assumptions}

				Assumes that the user is not trying to copy the same
				object into itself


		\end{description}


	\item{\textsc{getimageInfo}}
		\begin{description}
			\item{Purpose}
			
 				returns the width height and color depth 
				to reference variables

			\item{Input}

				\begin{itemize}
					\item{rows}

						This parameter grabs the number of rows
						in the imageType object

					\item{cols}

						This parameter grabs the number of cols
						in the imageType object

					\item{levels}

						This paremeter grabs the depth of the
						image in the imageType object

				\end{itemize}

			\item{Output}

				None

			\item{Assumptions}

				Assumes nothing but it makes sense that the object being
				queried has been loaded with some image


		\end{description}


	\item{\textsc{getpixelval}}
		\begin{description}
			\item{Purpose}

				Returns the value of a pixel

			\item{Input}

				\begin{itemize}

					\item{i}
					
						The row of the pixel

					\item{j} 

						The column of the pixel

				\end{itemize}

			\item{Output}

				The integer value of the pixel at 
				pixelValue[i][j]


			\item{Assumptions}

				It is assumed that the image has been intialized


		\end{description}


	\item{\textsc{setpixelval}}
		\begin{description}
			\item{Purpose}


			\item{Input}


			\item{Output}


			\item{Assumptions}


		\end{description}


	\item{\textsc{getsubimage}}
		\begin{description}
			\item{Purpose}


			\item{Input}


			\item{Output}


			\item{Assumptions}


		\end{description}


	\item{\textsc{meangray}}
		\begin{description}
			\item{Purpose}


			\item{Input}


			\item{Output}


			\item{Assumptions}


		\end{description}


	\item{\textsc{enlargeImage}}
		\begin{description}
			\item{Purpose}


			\item{Input}


			\item{Output}


			\item{Assumptions}


		\end{description}


	\item{\textsc{shrinkImage}}
		\begin{description}
			\item{Purpose}


			\item{Input}


			\item{Output}


			\item{Assumptions}


		\end{description}


	\item{\textsc{reflectImage}}
		\begin{description}
			\item{Purpose}


			\item{Input}


			\item{Output}


			\item{Assumptions}


		\end{description}


	\item{\textsc{translateImage}}
		\begin{description}
			\item{Purpose}


			\item{Input}


			\item{Output}


			\item{Assumptions}


		\end{description}


	\item{\textsc{rotateImage}}
		\begin{description}
			\item{Purpose}


			\item{Input}


			\item{Output}


			\item{Assumptions}


		\end{description}


	\item{\textsc{operator+}}
		\begin{description}
			\item{Purpose}


			\item{Input}


			\item{Output}


			\item{Assumptions}


		\end{description}


	\item{\textsc{operator-}}
		\begin{description}
			\item{Purpose}


			\item{Input}


			\item{Output}


			\item{Assumptions}


		\end{description}


	\item{\textsc{negateImage}}

\end{description}

\subsection{driver.cpp}

\begin{description}

	\item{\textsc{showmenu}}

		\begin{description}
			\item{Purpose}


			\item{Input}


			\item{Output}


			\item{Assumptions}


		\end{description}


	\item{\textsc{showregs}}

		\begin{description}
			\item{Purpose}


			\item{Input}


			\item{Output}


			\item{Assumptions}


		\end{description}


	\item{\textsc{drawwindow}}

		\begin{description}
			\item{Purpose}


			\item{Input}


			\item{Output}


			\item{Assumptions}


		\end{description}


	\item{\textsc{deletemenu}}

		\begin{description}
			\item{Purpose}


			\item{Input}


			\item{Output}


			\item{Assumptions}


		\end{description}


	\item{\textsc{deletewindow}}

		\begin{description}
			\item{Purpose}


			\item{Input}


			\item{Output}


			\item{Assumptions}


		\end{description}


	\item{\textsc{process entry}}
		\begin{description}
			\item{Purpose}


			\item{Input}


			\item{Output}


			\item{Assumptions}


		\end{description}



	\item{\textsc{stdwindow}}
		\begin{description}
			\item{Purpose}


			\item{Input}


			\item{Output}


			\item{Assumptions}


		\end{description}



	\item{\textsc{promptforreg}}
		\begin{description}
			\item{Purpose}


			\item{Input}


			\item{Output}


			\item{Assumptions}


		\end{description}



	\item{\textsc{promptforfilename}}
		\begin{description}
			\item{Purpose}


			\item{Input}


			\item{Output}


			\item{Assumptions}


		\end{description}



	\item{\textsc{promptforloc}}
		\begin{description}
			\item{Purpose}


			\item{Input}


			\item{Output}


			\item{Assumptions}


		\end{description}



	\item{\textsc{promptforpixvalue}}
		\begin{description}
			\item{Purpose}


			\item{Input}


			\item{Output}


			\item{Assumptions}


		\end{description}



	\item{\textsc{promptforscalevalue}}

		\begin{description}
			\item{Purpose}


			\item{Input}


			\item{Output}


			\item{Assumptions}


		\end{description}


	\item{\textsc{promptformirrow}}

		\begin{description}
			\item{Purpose}


			\item{Input}


			\item{Output}


			\item{Assumptions}


		\end{description}


	\item{\textsc{promptforangle}}

		\begin{description}
			\item{Purpose}


			\item{Input}


			\item{Output}


			\item{Assumptions}


		\end{description}


	\item{\textsc{messagebox}}
		\begin{description}
			\item{Purpose}


			\item{Input}


			\item{Output}


			\item{Assumptions}


		\end{description}



	\item{\textsc{fillregs}}
		\begin{description}
			\item{Purpose}


			\item{Input}


			\item{Output}


			\item{Assumptions}


		\end{description}



	\item{\textsc{clearregisters}}
		\begin{description}
			\item{Purpose}


			\item{Input}


			\item{Output}


			\item{Assumptions}


		\end{description}



	\item{\textsc{laodimage}}
		\begin{description}
			\item{Purpose}


			\item{Input}


			\item{Output}


			\item{Assumptions}


		\end{description}



	\item{\textsc{saveimage}}
		\begin{description}
			\item{Purpose}


			\item{Input}


			\item{Output}


			\item{Assumptions}


		\end{description}



	\item{\textsc{getimage}}
		\begin{description}
			\item{Purpose}


			\item{Input}


			\item{Output}


			\item{Assumptions}


		\end{description}



	\item{\textsc{setpixel}}
		\begin{description}
			\item{Purpose}


			\item{Input}


			\item{Output}


			\item{Assumptions}


		\end{description}



	\item{\textsc{getpixel}}
		\begin{description}
			\item{Purpose}


			\item{Input}


			\item{Output}


			\item{Assumptions}


		\end{description}



	\item{\textsc{extractsub}}
		\begin{description}
			\item{Purpose}


			\item{Input}


			\item{Output}


			\item{Assumptions}


		\end{description}



	\item{\textsc{enlargeimg}}
		\begin{description}
			\item{Purpose}


			\item{Input}


			\item{Output}


			\item{Assumptions}


		\end{description}



	\item{\textsc{shrinkimg}}
		\begin{description}
			\item{Purpose}


			\item{Input}


			\item{Output}


			\item{Assumptions}


		\end{description}



	\item{\textsc{reflectimg}}
		\begin{description}
			\item{Purpose}


			\item{Input}


			\item{Output}


			\item{Assumptions}


		\end{description}



	\item{\textsc{translateimg}}
		\begin{description}
			\item{Purpose}


			\item{Input}


			\item{Output}


			\item{Assumptions}


		\end{description}



	\item{\textsc{rotateimg}}
		\begin{description}
			\item{Purpose}


			\item{Input}


			\item{Output}


			\item{Assumptions}


		\end{description}



	\item{\textsc{sumimg}}
		\begin{description}
			\item{Purpose}


			\item{Input}


			\item{Output}


			\item{Assumptions}


		\end{description}



	\item{\textsc{subtractimg}}
		\begin{description}
			\item{Purpose}


			\item{Input}


			\item{Output}


			\item{Assumptions}


		\end{description}



	\item{\textsc{negateimg}}
		\begin{description}
			\item{Purpose}


			\item{Input}


			\item{Output}


			\item{Assumptions}


		\end{description}



	\item{\textsc{findlocalpgm}}
		\begin{description}
			\item{Purpose}


			\item{Input}


			\item{Output}


			\item{Assumptions}


		\end{description}


\end{description}

\subsection{cubicspline.h}

\begin{description}
	\item{\textsc{constructor}}
		\begin{description}
			\item{Purpose}


			\item{Input}


			\item{Output}


			\item{Assumptions}


		\end{description}


	\item{\textsc{copy constructor}}
		\begin{description}
			\item{Purpose}


			\item{Input}


			\item{Output}


			\item{Assumptions}


		\end{description}


	\item{\textsc{constructor with parameters}}
		\begin{description}
			\item{Purpose}


			\item{Input}


			\item{Output}


			\item{Assumptions}


		\end{description}


	\item{\textsc{destructor}}
		\begin{description}
			\item{Purpose}


			\item{Input}


			\item{Output}


			\item{Assumptions}


		\end{description}


	\item{\textsc{create}}
		\begin{description}
			\item{Purpose}


			\item{Input}


			\item{Output}


			\item{Assumptions}


		\end{description}


	\item{\textsc{createcubic}}
		\begin{description}
			\item{Purpose}


			\item{Input}


			\item{Output}


			\item{Assumptions}


		\end{description}


	\item{\textsc{getval}}
		\begin{description}
			\item{Purpose}


			\item{Input}


			\item{Output}


			\item{Assumptions}


		\end{description}


	\item{\textsc{getcubicval}}
		\begin{description}
			\item{Purpose}


			\item{Input}


			\item{Output}


			\item{Assumptions}


		\end{description}


\end{description}

\subsection{imageIO.h}

\begin{description}
	\item{\textsc{readimageheader}}
		\begin{description}
			\item{Purpose}


			\item{Input}


			\item{Output}


			\item{Assumptions}


		\end{description}


	\item{\textsc{readimage}}
		\begin{description}
			\item{Purpose}


			\item{Input}


			\item{Output}


			\item{Assumptions}


		\end{description}


	\item{\textsc{writeimage}}
		\begin{description}
			\item{Purpose}


			\item{Input}


			\item{Output}


			\item{Assumptions}


		\end{description}


\end{description}

\subsection{comp\_curses.h}
\begin{description}

	\item{\textsc{startcurses}}
		\begin{description}
			\item{Purpose}


			\item{Input}


			\item{Output}


			\item{Assumptions}


		\end{description}


	\item{\textsc{endcurses}}
		\begin{description}
			\item{Purpose}


			\item{Input}


			\item{Output}


			\item{Assumptions}


		\end{description}


	\item{\textsc{setcolor}}
		\begin{description}
			\item{Purpose}


			\item{Input}


			\item{Output}


			\item{Assumptions}


		\end{description}


	\item{\textsc{screenwidth}}
		\begin{description}
			\item{Purpose}


			\item{Input}


			\item{Output}


			\item{Assumptions}


		\end{description}


	\item{\textsc{screenheigth}}
		\begin{description}
			\item{Purpose}


			\item{Input}


			\item{Output}


			\item{Assumptions}


		\end{description}


	\item{\textsc{promptforint}}
		\begin{description}
			\item{Purpose}


			\item{Input}


			\item{Output}


			\item{Assumptions}


		\end{description}


	\item{\textsc{promptfordouble}}
		\begin{description}
			\item{Purpose}


			\item{Input}


			\item{Output}


			\item{Assumptions}


		\end{description}


	\item{\textsc{promptforstring}}
		\begin{description}
			\item{Purpose}


			\item{Input}


			\item{Output}


			\item{Assumptions}


		\end{description}


\end{description}

\section{Bugs and Errors}

hmm what goes here

\section{What was Learned}

lol

\section{Division of Labor}

ok!

\end{document}
