\documentclass[pdftex, 11pt]{article}

\usepackage[pdftex]{graphicx}
\usepackage{listings}
\usepackage{fancyhdr}
\usepackage{lastpage}
\usepackage{fullpage}
\usepackage{color}
\usepackage[pdftex,bookmarks=true,colorlinks=true,linkcolor=blue]{hyperref}

%this sets the line at the header
\setlength{\headheight}{15.2pt}

\newcommand{\HRule}{\rule{\linewidth}{0.5mm}}

% This is all for formatting and making the Table of Contents according to 
% spec. Don't play with it.
\makeatletter
\renewcommand\l@section[2]{%
  \ifnum \c@tocdepth >\z@
    \addpenalty\@secpenalty
    \addvspace{1.0em \@plus\p@}%
    \setlength\@tempdima{1.5em}%
    \begingroup
      \parindent \z@ \rightskip \@pnumwidth
      \parfillskip -\@pnumwidth
      \leavevmode \bfseries
      \advance\leftskip\@tempdima
      \hskip -\leftskip
      #1\nobreak\ 
      \leaders\hbox{$\m@th\mkern \@dotsep mu\hbox{.}\mkern \@dotsep mu$}
     \hfil \nobreak\hb@xt@\@pnumwidth{\hss #2}\par
    \endgroup
  \fi}
\makeatother



\begin{document}
%import the title page
\begin{titlepage}
	\begin{center}

		% Upper part of the page
		\textsc{\LARGE University of Nevada, Reno}\\[.5cm]
		\includegraphics[width=0.15\textwidth]{./logo.png}\\[.5cm]

		\textsc{\large CS 302 | Data Structures } \\[.5cm]

		% Title
		\HRule \\[0.4cm]
		{ \huge \bfseries Assignment \#5}\\[0.4cm]

		\HRule \\[1.5cm]

		% Author and supervisor
		\begin{minipage}{0.4\textwidth}
			\begin{flushleft} \large
				\emph{Students:}\\
				Joshua \textsc{Gleason}\\
				Josiah \textsc{Humphrey}
			\end{flushleft}
		\end{minipage}
		\begin{minipage}{0.4\textwidth}
			\begin{flushright} \large
				\emph{Instructor:} \\
				Dr. George \textsc{Bebis}
			\end{flushright}
		\end{minipage}

		\vfill

		% Bottom of the page
		{\large \today}

	\end{center}

\end{titlepage}


%headers, footers, and table of contents
\pagestyle{fancy}
\renewcommand{\sectionmark}[1]{\markright{\thesection}}
\rhead{Page \thepage\ of \pageref{LastPage}}
\lhead{}
\lfoot{CS 302 | Spring 2010}
\cfoot{}
\renewcommand{\footrulewidth}{0.4pt}

\tableofcontents
\newpage

\lhead{Joshua Gleason \& Josiah Humphrey}
\rhead{Page \thepage\ of \pageref{LastPage}}
\rfoot{Section\ \rightmark}
%\rfoot{Section\ \if\relax\thesubsection\relax \rightmark \else \thesection\fi}
\cfoot{}
\lfoot{CS 302 | Spring 2010}
\renewcommand{\footrulewidth}{0.4pt}

\section{Introduction}


\section{Use of Code}


\section{Functions}

\subsection{Image.h}
\begin{description}

	\item{\textsc{constructor}}
		\begin{description}
			\item{Purpose}

				default constructor allocates no memory and sets the size to zero 

			\item{Input}

				None

			\item{Output}

				None

			\item{Assumptions}

				Sets everything to zero and 
				sets the pixelValue array to NULL

		\end{description}


	\item{\textsc{constructor with parameters}}
		\begin{description}
			\item{Purpose}

				change the dimenstions of the image, delete,
				and re-allocate memory if required

			\item{Input}

				An N, M, and Q value to set the new image to

			\item{Output}

				None

			\item{Assumptions}

				Sets the image to a certain size and intializes the
				image as a grid

		\end{description}



	\item{\textsc{desctructor}}
		\begin{description}
			\item{Purpose}

				Deletes and memory that has been dynamically allocated

			\item{Input}

				None

			\item{Output}

				None

			\item{Assumptions}

				Checks to see if the pixelValue array has been set
				if so, deletes

		\end{description}


	\item{\textsc{copy\_constructor}}
		\begin{description}
			\item{Purpose}
		
				Creates a new array absed on the thing to be copied
				then sets the pixelValue of the new object the same as
				the old image

			\item{Input}

				ImageType rhs is the old image to be copied over into
				the new array


			\item{Output}

				None

			\item{Assumptions}

				The old image must be passed as reference to prevent
				an infinate loop


		\end{description}


	\item{\textsc{operator=}}
		\begin{description}
			\item{Purpose}

				equal operator overload, this is basically
				the same as the copy constructor
				except it will likely have to 
				de-allocate memory before copying values, all
				this is decided in setImageInfo however

			\item{Input}

				imageType rhs which is the old iamge to be 
				copied over to the new image

			\item{Output}

				Returns the imageType obejct so that
				equal chaining can be implemented


			\item{Assumptions}

				Assumes that the user is not trying to copy the same
				object into itself


		\end{description}


	\item{\textsc{getimageInfo}}
		\begin{description}
			\item{Purpose}
			
 				returns the width height and color depth 
				to reference variables

			\item{Input}

				\begin{itemize}
					\item{rows}

						This parameter grabs the number of rows
						in the imageType object

					\item{cols}

						This parameter grabs the number of cols
						in the imageType object

					\item{levels}

						This paremeter grabs the depth of the
						image in the imageType object

				\end{itemize}

			\item{Output}

				None

			\item{Assumptions}

				Assumes nothing but it makes sense that the object being
				queried has been loaded with some image


		\end{description}


	\item{\textsc{setImageInfo}}
		\begin{description}
			\item{Purpose}
			
				Sets the image info, deleting and allocating memory
				as required, also creates a background grid

			\item{Input}

				\begin{itemize}
					\item{rows}

						This parameter sets the number of rows
						in the imageType object

					\item{cols}

						This parameter sets the number of cols
						in the imageType object

					\item{levels}

						This paremeter sets the depth of the
						image in the imageType object

				\end{itemize}

			\item{Output}

				None

			\item{Assumptions}

				Assumes nothing

		\end{description}




	\item{\textsc{getpixelval}}
		\begin{description}
			\item{Purpose}

				Returns the value of a pixel

			\item{Input}

				\begin{itemize}

					\item{i}
					
						The row of the pixel

					\item{j} 

						The column of the pixel

				\end{itemize}

			\item{Output}

				The integer value of the pixel at 
				pixelValue[i][j]


			\item{Assumptions}

				It is assumed that the image has been intialized


		\end{description}


	\item{\textsc{setpixelval}}
		\begin{description}
			\item{Purpose}

				Sets the value of a pixel

			\item{Input}

				\begin{itemize}

					\item{i}

						The row of the pixel to be changed

					\item{j}

						The column of the pixel to be changed

				\end{itemize}

			\item{Output}

				None

			\item{Assumptions}

				Assumes the image has been intialized

		\end{description}


	\item{\textsc{getsubimage}}
		\begin{description}
			\item{Purpose}

 				Obtain a sub-image from old.  Uses the coordinates
				of the upper left corner
				and lower right corner to obtain image.

			\item{Input}

				\begin{itemize}

					\item{ULr}
	
						The upper left row of the pixel
						to be x in (0,0) in the new image.

					\item{ULc}

						The upper left column of the pixel
						to be y in (0,0) in the new image

					\item{LRr}

						The lower right row of the pixel
						to be x in (max\_x, max\_y) in the
						new image

					\item{LRC}

						The lower right row of the pixel
						to be y in (max\_x, max\_y) in the
						new image

				\end{itemize}


			\item{Output}

				None


			\item{Assumptions}

				Assumes that the UL\{r,c\} and LR\{r,c\} have been
				properly bounds and error checked before the function
				call


		\end{description}


	\item{\textsc{meangray}}
		\begin{description}
			\item{Purpose}

				
				this calculates the average gray value in the
				picture, this is done by adding
				all of the pixels and dividing by the total
				number of pixels

			\item{Input}

				None

			\item{Output}

				A double value that is the mean value of all
				the pixels in pixelValue


			\item{Assumptions}

				Assumes nothing and returns 0 if the image 
				has not been intialized


		\end{description}


	\item{\textsc{enlargeImage}}
		\begin{description}
			\item{Purpose}
				
				This function enlarges an image by a 
				magnitude of s, so for example if the
				original function was 100x100 and s is 
				10, then the new image is 1000x1000

			\item{Input}

				\begin{itemize}

					\item{S}

						This is the magnitude of the enlargement

						The function is also overloaded 
						to accept ints as well as doubles

					\item{ImageType old}

						This is the image to be enlarged

					\item{cubic}

						A bool value that decides which type of
						interpolation to use.
						If true, use cubic interpolation
						If false, use linear interpolation
				\end{itemize}

			\item{Output}

				None

			\item{Assumptions}


				The method choosen to use was bicubic/linear
				interpolation which creates
				splines for each column(cubic or linear),
				then using those splines create an
				image which is a stretched version of the original 
				image.  The way this was achieved
				was to stretch the entire image only vertically,
				and then stretch that
				image horizontally.  Then the same thing was done except
				reversed (stretched image
				horizontally first) and then the two image summed 
				together.  This gives an
				average value between both methods.  Although it can
				handle S values less
				than 1, the shrinkImage function works better for this.



		\end{description}

	\item{\textsc{shrinkImage}}
		\begin{description}
			\item{Purpose}

				Shrink image, average all the values
				in the block to make the new pixel, this
				makes the shrink much less jagged looking in the end

			\item{Input}

				\begin{itemize}

					\item{s}

						The inteter value of the shrink factor

					\item{ImageType old}

						The image to be shrunk

				\end{itemize}

			\item{Output}

				None

			\item{Assumptions}

				Assumes the image has been intialized and that error
				checking has been done.

		\end{description}


	\item{\textsc{reflectImage}}
		\begin{description}
			\item{Purpose}

				reflects image by moving the pixel to N or M
				minus the current row or column
				depending on the value of the flag
				(true being a horizontal reflection and
				false being a vertical reflection)

			\item{Input}

				\begin{itemize}

					\item{flag}

						The flag that sets either vertical or
						horizontal reflection

					\item{ImageType old}

						The image to be reflected

				\end{itemize}

			\item{Output}

				None

			\item{Assumptions}

				Assumes nothing, but it makes sense to have an intialized
				image to reflect

		\end{description}


	\item{\textsc{translateImage}}
		\begin{description}
			\item{Purpose}

				Translate the image down to the right,
				any part that goes out of the screen is
 				not calculated.  Checkered background from
				setImageInfo is retained.

			\item{Input}

				\begin{itemize}

					\item{t}

						The integer value of the translation. The
						translation will occur down and 
						to the right 't' pixels

					\item{ImageType old}

						The image to be translated

				\end{itemize}

			\item{Output}

				None

			\item{Assumptions}

				No assumptions are made, but it makes sense to have
				an intialized image

		\end{description}


	\item{\textsc{rotateImage}}
		\begin{description}
			\item{Purpose}

				Rotate the image clockwise using bilinear
				interpolation, basically traversing
				the entire image going from the destination 
				to the source by using the
				in reverse (which is why its clockwise). 
				Once a location is determined the
				surrounding pixels are used to calculate 
				intermediate values between the
				pixels, this gives a pretty smooth rotate.

			\item{Input}

				\begin{itemize}

					\item{theta}

						The degrees to rotate. This is converted
						to radians inside the function

					\item{ImnageType old}

						The image to be rotated

				\end{itemize}

			\item{Output}

				None

			\item{Assumptions}

				Assumes that theta is in degrees because theta is
				converted to radians from degrees inside the function
				for the use of the trig functions.
				It is also assumed that the image has been intialized
				before the function call. It is also assumed that theta
				is between 0 and 360.

		\end{description}


	\item{\textsc{operator+}}
		\begin{description}
			\item{Purpose}

				
				Sum two images together, basically just finding
				the average pixel value of
				every pixel between two images.  Throws an
				exception if dimesions of both
				images don't match

			\item{Input}

				\begin{itemize}
						
					\item{ImageType rhs}

						This is the image to be added to
						'this' image

				\end{itemize}

			\item{Output}

				ImageType object to chain additions

			\item{Assumptions}

				It is assumed that each image have the same dimensions.
				However, if the images do not have the same dimensions,
				then a string is thrown stating that the images do
				not have the same dimensions. It is not neccesary to have
				each image initialized, but it makes senses
				that they would each be initialized.

		\end{description}


	\item{\textsc{operator-}}
		\begin{description}
			\item{Purpose}

				subtract two images from each other to see the 
				differences, if the magnitude of
				the difference is less then Q/6 then the pixel
				is replaced with black,	otherwise white is used. 
				This seems to help reduce 
				the amount of noise in the pictures

			\item{Input}

				\begin{itemize}

					\item{ImageType rhs}

						This is the image to be subtracted from
						'this' image

				\end{itemize}

			\item{Output}

				ImageType is returned to allow chaining of subtraction

			\item{Assumptions}

				It is assumed that each image have the same dimensions.
				However, if the images do not have the same dimensions,
				then a string is thrown stating that the images do
				not have the same dimensions. It is not neccesary to have
				each image initialized, but it makes senses
				that they would each be initialized.

		\end{description}


	\item{\textsc{negateImage}}

\end{description}

\subsection{driver.cpp}

\begin{description}

	\item{\textsc{showmenu}}

		\begin{description}
			\item{Purpose}


			\item{Input}


			\item{Output}


			\item{Assumptions}


		\end{description}


	\item{\textsc{showregs}}

		\begin{description}
			\item{Purpose}


			\item{Input}


			\item{Output}


			\item{Assumptions}


		\end{description}


	\item{\textsc{drawwindow}}

		\begin{description}
			\item{Purpose}


			\item{Input}


			\item{Output}


			\item{Assumptions}


		\end{description}


	\item{\textsc{deletemenu}}

		\begin{description}
			\item{Purpose}


			\item{Input}


			\item{Output}


			\item{Assumptions}


		\end{description}


	\item{\textsc{deletewindow}}

		\begin{description}
			\item{Purpose}


			\item{Input}


			\item{Output}


			\item{Assumptions}


		\end{description}


	\item{\textsc{process entry}}
		\begin{description}
			\item{Purpose}


			\item{Input}


			\item{Output}


			\item{Assumptions}


		\end{description}



	\item{\textsc{stdwindow}}
		\begin{description}
			\item{Purpose}


			\item{Input}


			\item{Output}


			\item{Assumptions}


		\end{description}



	\item{\textsc{promptforreg}}
		\begin{description}
			\item{Purpose}


			\item{Input}


			\item{Output}


			\item{Assumptions}


		\end{description}



	\item{\textsc{promptforfilename}}
		\begin{description}
			\item{Purpose}


			\item{Input}


			\item{Output}


			\item{Assumptions}


		\end{description}



	\item{\textsc{promptforloc}}
		\begin{description}
			\item{Purpose}


			\item{Input}


			\item{Output}


			\item{Assumptions}


		\end{description}



	\item{\textsc{promptforpixvalue}}
		\begin{description}
			\item{Purpose}


			\item{Input}


			\item{Output}


			\item{Assumptions}


		\end{description}



	\item{\textsc{promptforscalevalue}}

		\begin{description}
			\item{Purpose}


			\item{Input}


			\item{Output}


			\item{Assumptions}


		\end{description}


	\item{\textsc{promptformirrow}}

		\begin{description}
			\item{Purpose}


			\item{Input}


			\item{Output}


			\item{Assumptions}


		\end{description}


	\item{\textsc{promptforangle}}

		\begin{description}
			\item{Purpose}


			\item{Input}


			\item{Output}


			\item{Assumptions}


		\end{description}


	\item{\textsc{messagebox}}
		\begin{description}
			\item{Purpose}


			\item{Input}


			\item{Output}


			\item{Assumptions}


		\end{description}



	\item{\textsc{fillregs}}
		\begin{description}
			\item{Purpose}


			\item{Input}


			\item{Output}


			\item{Assumptions}


		\end{description}



	\item{\textsc{clearregisters}}
		\begin{description}
			\item{Purpose}


			\item{Input}


			\item{Output}


			\item{Assumptions}


		\end{description}



	\item{\textsc{laodimage}}
		\begin{description}
			\item{Purpose}


			\item{Input}


			\item{Output}


			\item{Assumptions}


		\end{description}



	\item{\textsc{saveimage}}
		\begin{description}
			\item{Purpose}


			\item{Input}


			\item{Output}


			\item{Assumptions}


		\end{description}



	\item{\textsc{getimage}}
		\begin{description}
			\item{Purpose}


			\item{Input}


			\item{Output}


			\item{Assumptions}


		\end{description}



	\item{\textsc{setpixel}}
		\begin{description}
			\item{Purpose}


			\item{Input}


			\item{Output}


			\item{Assumptions}


		\end{description}



	\item{\textsc{getpixel}}
		\begin{description}
			\item{Purpose}


			\item{Input}


			\item{Output}


			\item{Assumptions}


		\end{description}



	\item{\textsc{extractsub}}
		\begin{description}
			\item{Purpose}


			\item{Input}


			\item{Output}


			\item{Assumptions}


		\end{description}



	\item{\textsc{enlargeimg}}
		\begin{description}
			\item{Purpose}


			\item{Input}


			\item{Output}


			\item{Assumptions}


		\end{description}



	\item{\textsc{shrinkimg}}
		\begin{description}
			\item{Purpose}


			\item{Input}


			\item{Output}


			\item{Assumptions}


		\end{description}



	\item{\textsc{reflectimg}}
		\begin{description}
			\item{Purpose}


			\item{Input}


			\item{Output}


			\item{Assumptions}


		\end{description}



	\item{\textsc{translateimg}}
		\begin{description}
			\item{Purpose}


			\item{Input}


			\item{Output}


			\item{Assumptions}


		\end{description}



	\item{\textsc{rotateimg}}
		\begin{description}
			\item{Purpose}


			\item{Input}


			\item{Output}


			\item{Assumptions}


		\end{description}



	\item{\textsc{sumimg}}
		\begin{description}
			\item{Purpose}


			\item{Input}


			\item{Output}


			\item{Assumptions}


		\end{description}



	\item{\textsc{subtractimg}}
		\begin{description}
			\item{Purpose}


			\item{Input}


			\item{Output}


			\item{Assumptions}


		\end{description}



	\item{\textsc{negateimg}}
		\begin{description}
			\item{Purpose}


			\item{Input}


			\item{Output}


			\item{Assumptions}


		\end{description}



	\item{\textsc{findlocalpgm}}
		\begin{description}
			\item{Purpose}


			\item{Input}


			\item{Output}


			\item{Assumptions}


		\end{description}


\end{description}

\subsection{cubicspline.h}

\begin{description}
	\item{\textsc{constructor}}
		\begin{description}
			\item{Purpose}


			\item{Input}


			\item{Output}


			\item{Assumptions}


		\end{description}


	\item{\textsc{copy constructor}}
		\begin{description}
			\item{Purpose}


			\item{Input}


			\item{Output}


			\item{Assumptions}


		\end{description}


	\item{\textsc{constructor with parameters}}
		\begin{description}
			\item{Purpose}


			\item{Input}


			\item{Output}


			\item{Assumptions}


		\end{description}


	\item{\textsc{destructor}}
		\begin{description}
			\item{Purpose}


			\item{Input}


			\item{Output}


			\item{Assumptions}


		\end{description}


	\item{\textsc{create}}
		\begin{description}
			\item{Purpose}


			\item{Input}


			\item{Output}


			\item{Assumptions}


		\end{description}


	\item{\textsc{createcubic}}
		\begin{description}
			\item{Purpose}


			\item{Input}


			\item{Output}


			\item{Assumptions}


		\end{description}


	\item{\textsc{getval}}
		\begin{description}
			\item{Purpose}


			\item{Input}


			\item{Output}


			\item{Assumptions}


		\end{description}


	\item{\textsc{getcubicval}}
		\begin{description}
			\item{Purpose}


			\item{Input}


			\item{Output}


			\item{Assumptions}


		\end{description}


\end{description}

\subsection{imageIO.h}

\begin{description}
	\item{\textsc{readimageheader}}
		\begin{description}
			\item{Purpose}


			\item{Input}


			\item{Output}


			\item{Assumptions}


		\end{description}


	\item{\textsc{readimage}}
		\begin{description}
			\item{Purpose}


			\item{Input}


			\item{Output}


			\item{Assumptions}


		\end{description}


	\item{\textsc{writeimage}}
		\begin{description}
			\item{Purpose}


			\item{Input}


			\item{Output}


			\item{Assumptions}


		\end{description}


\end{description}

\subsection{comp\_curses.h}
\begin{description}

	\item{\textsc{startcurses}}
		\begin{description}
			\item{Purpose}

				This initializes the curses screen and its functions

			\item{Input}

				None

			\item{Output}

				None

			\item{Assumptions}

				No assumptions are made besides have a terminal
				capable of displaying curses correctly.

		\end{description}


	\item{\textsc{endcurses}}
		\begin{description}
			\item{Purpose}

				This ends the curses screen and its functions

			\item{Input}

				None

			\item{Output}

				None

			\item{Assumptions}

				This assumes that curses has ibeen initialized
				with startCurses()


		\end{description}


	\item{\textsc{setcolor}}
		\begin{description}
			\item{Purpose}

				This sets the colors for stdscr

			\item{Input}

				\begin{itemize}

					\item{*somewin}

						This is the window pointer to set
						the colors to a specific window

					\item{cf}

						This is the first color (foreground)
						for the color pair to set in the window

					\item{cb}

						This is the second colod (background)
						for the color pair to set in the window
						
				\end{itemize}	

			\item{Output}

				None

			\item{Assumptions}

				Assumes that screen has been initialized

		\end{description}


	\item{\textsc{screenwidth}}
		\begin{description}
			\item{Purpose}

				Returns the max screen x value 

			\item{Input}

				None

			\item{Output}

				The int value of the max x value for the entire terminal

			\item{Assumptions}

				Assumes startCurses() has been run

		\end{description}


	\item{\textsc{screenheigth}}
		\begin{description}
			\item{Purpose}

				Returns the max screen y value

			\item{Input}

				None

			\item{Output}

				The int value of the max y value for the entire terminal

			\item{Assumptions}

				Assumes startCurses() has been run

		\end{description}


	\item{\textsc{promptforint}}
		\begin{description}
			\item{Purpose}

				Prompts for an int at some int at some (x,y) cordinate

			\item{Input}

				\begin{itemize}

					\item{*somewin}

						Some window to prompt for the int in

					\item{y}

						The y cordinate at which to prompt
						for the int

					\item{x} 

						The x cordinate at which to prompt
						for the int

					\item{promptString[]}

						The string to display when prompting for
						the int
						
				\end{itemize}

			\item{Output}

				The integer value of the user's input

			\item{Assumptions}

				It is assumed that startCurses() has been run.
				The function has built in error checking to prevent
				bad data from being input

		\end{description}


	\item{\textsc{promptfordouble}}
		\begin{description}
			\item{Purpose}

				Prompts for a double at some int at some (x,y) cordinate

			\item{Input}

				\begin{itemize}

					\item{*somewin}

						Some window to prompt for the double in

					\item{y}

						The y cordinate at which to prompt
						for the double

					\item{x} 

						The x cordinate at which to prompt
						for the double

					\item{promptString[]}

						The string to display when prompting for
						the double
						
				\end{itemize}



			\item{Output}

				The double value of the user's input

			\item{Assumptions}

				It is assumed that startCurses() has been run.
				The function has built in error checking to prevent
				bad data from being input (such as multiple periods)

		\end{description}


	\item{\textsc{promptforstring}}
		\begin{description}
			\item{Purpose}

				Prompts for a string at some (x,y) cordinate

			\item{Input}

				\begin{itemize}

					\item{*somewin}

						The window at which to prompt
						for the string

					\item{y}

						The y cordinate at which to prompt

					\item{x}

						The rxy cordinate at which to prompt

					\item{promptstring}

						The string to display when prompting
						for the string

					\item{str[]}

						The array for the string that is typed
						in by the user

					\item{len}

						The length of the string stored

				\end{itemize}

			\item{Output}

				None

			\item{Assumptions}

				It is assumed that startCurses() has been run.
				The function also accounts for backspaces and
				makes sure that only valid input is entered.

		\end{description}


\end{description}

\section{Bugs and Errors}

hmm what goes here

\section{What was Learned}

lol

\section{Division of Labor}

ok!

\end{document}
