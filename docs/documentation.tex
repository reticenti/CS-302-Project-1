\documentclass[pdftex, 11pt]{article}

\usepackage[pdftex]{graphicx}
\usepackage{listings}
\usepackage{fancyhdr}
\usepackage{lastpage}
\usepackage{fullpage}
\usepackage{color}
\usepackage{amsmath}
\usepackage[pdftex,bookmarks=true,colorlinks=true,linkcolor=blue]{hyperref}
\usepackage{subfig}

%this sets the line at the header
\setlength{\headheight}{15.2pt}

\newcommand{\HRule}{\rule{\linewidth}{0.5mm}}

% This is all for formatting and making the Table of Contents according to 
% spec. Don't play with it.
\makeatletter
\renewcommand\l@section[2]{%
  \ifnum \c@tocdepth >\z@
    \addpenalty\@secpenalty
    \addvspace{1.0em \@plus\p@}%
    \setlength\@tempdima{1.5em}%
    \begingroup
      \parindent \z@ \rightskip \@pnumwidth
      \parfillskip -\@pnumwidth
      \leavevmode \bfseries
      \advance\leftskip\@tempdima
      \hskip -\leftskip
      #1\nobreak\ 
      \leaders\hbox{$\m@th\mkern \@dotsep mu\hbox{.}\mkern \@dotsep mu$}
     \hfil \nobreak\hb@xt@\@pnumwidth{\hss #2}\par
    \endgroup
  \fi}
\makeatother



\begin{document}

%import the title page
\begin{titlepage}
	\begin{center}

		% Upper part of the page
		\textsc{\LARGE University of Nevada, Reno}\\[.5cm]
		\includegraphics[width=0.15\textwidth]{./logo.png}\\[.5cm]

		\textsc{\large CS 302 | Data Structures } \\[.5cm]

		% Title
		\HRule \\[0.4cm]
		{ \huge \bfseries Assignment \#5}\\[0.4cm]

		\HRule \\[1.5cm]

		% Author and supervisor
		\begin{minipage}{0.4\textwidth}
			\begin{flushleft} \large
				\emph{Students:}\\
				Joshua \textsc{Gleason}\\
				Josiah \textsc{Humphrey}
			\end{flushleft}
		\end{minipage}
		\begin{minipage}{0.4\textwidth}
			\begin{flushright} \large
				\emph{Instructor:} \\
				Dr. George \textsc{Bebis}
			\end{flushright}
		\end{minipage}

		\vfill

		% Bottom of the page
		{\large \today}

	\end{center}

\end{titlepage}


%headers, footers, and table of contents
\pagestyle{fancy}
\renewcommand{\sectionmark}[1]{\markright{\thesection}}
\rhead{Page \thepage\ of \pageref{LastPage}}
\lhead{}
\lfoot{CS 302 | Spring 2010}
\cfoot{}
\renewcommand{\footrulewidth}{0.4pt}

\tableofcontents

\listoffigures
\newpage

\lhead{Joshua Gleason \& Josiah Humphrey}
\rhead{Page \thepage\ of \pageref{LastPage}}
\rfoot{Section\ \rightmark}
\cfoot{}
\lfoot{CS 302 | Spring 2010}
\renewcommand{\footrulewidth}{0.4pt}

\section{Introduction}

In this programming assignment, we continued and built upon image processing. We took out previous programming
assignment and added more functionality. Our goals were to represent the regions found by connected components in a list
and to compute a number of useful properties to characterize their position, orientation, shape, and intensity. This was
a challenge that tested our ability to understand the data structures and image processing. We were also asked to use our
skills to manipulate items in lists, use templates, and learn about feature extraction and classification. 

In this assignment, we learned about geometric properties of objects and how to objectively classify them using
mathematical formulas. We were able to implement the equations that found various geometric properties of objects found
in images. Since this is a introduction to image processing, we were only asked to implement geometric and intensity
properties only. Using more advanced techniques we could have done more complicated things.

To calculate these properties, we used a technique from probability theory called moments. A moment is defined as 
\[
M_{p,q} = \sum_{i,j \in R} i^{p} j^{q}
\]

This was the basis for many of our calculations and was implemented as a generic function that would take any numbers
for $p$ and $q$.

\section{Use of Code}

The use of this code should be fairly straight forward. The main menu is easy to understand, just select the option you
want and hit enter. To load images into the programs image registers, you can either do it with the menu option, or on
the command line. To load images on the command line, just enter them as arguments. For instance, you could do \texttt{\$
./main.out images/hubble1.pgm}. You can enter as many images as you want on the command line like so: \texttt{\$
./main.out images/hubble1.pgm images/hubble2.pgm images/hubble3.pgm}

To classify the regions in an image, you need to choose the classify region option in the menu. Once you do this, you
will be able to choose which of the classifiers you would like to do. You can do multiple things to the same image. Once
you are done classifying the image, make sure you choose the save image option that is in that sub-menu to save all of
the changes. You will then need to save the image at the top of the main menu in order to output the image. 

One note about the classifiers. There are minimum and maximum values for each function, but to make them easier to use,
if the user inputs bounds that are below the minimum and/or above the maximum, it defaults to the minimum and/or maximum
value respectively. This was done to create a more user friendly program that is capable of accepting data that the
user intends. For instance, if the user wants to include the maximum size, you can just enter "99999999" and it will
interpret that input as a maximum value, but the program will automatically make that input sane for the environment.

\section{Functions}

\subsection{Image.h}

\begin{description}

		\item{\textsc{dilate}}
			\begin{description}

	\begin{lstlisting}
	void ImageType<pType>::dilate()
	\end{lstlisting}

				\item{Purpose}

					This function changes any pixel to black that is
					touching a black pixel on any of its 8 sides.

				\item{Input}

					None

				\item{Output}

					None

				\item{Assumptions}

					Nothing is assumed but it makes sense to
					actually have a defined picture.

			\end{description}

\end{description}

\section{Bugs and Errors}

During the creating of this program, there was one single bug that took a very, very long time to track down, following
is a detailed explanation of the bug and the methods used to track down and repair it.

The problem originally manifested itself as a segmentation fault when the choice to 'Classify Regions' was selected in
the main menu.  At first I looked through the classifyRegions function for any obvious problems, after that search came
up empty I began using the GDB debugger to track down the fatal error.

The first thing I needed to know was where the actual error was occurring, so I executed the program in GDB.  After the
re-creating the segmentation fault I found that the crash was occurring a conditional statement inside of the $==$
operator overload function inside of the Region Type class.  By examining parameters passed to the function I discovered
that the right hand side was actually an invalid value, printing the address of the parameter I found the value was
actually NULL.  This seemed very strange, so I used GDB's backtrace command, which indicated that the comparison was
taking place in the deleteItem function of the sortedList class or more specific the list of regions for the image.

Before debugging further, I pondered the recently acquired information and came to a hypothesis.  I believed that the
deleteItem function was not finding the value that it was passed even thought the RegionType values were being directly
taken from the list of regions.  This was the only way I could conceive the $==$ operator being passed NULL from
deleteItem.  Some more debugging was definitely needed to verify this claim and also answer some other questions if this
was the case.

After setting a breakpoint in the deleteItem function I ran the program and selected the Classify Regions option.  The
program paused at the first breakpoint where I obtained some very interesting information about the RegionType in
question.  I ran the command \texttt{print *this} in GDB so that I could quickly see all of the private members of the
current object.  To my surprise a few of the values were definitely invalid values, which may explain why the $==$
operator never returned true, even if the values had the same data members.  What would happen if you tried to compare
two invalid double values, even if they were copies of one another?  I had to determine the answer to this question.  By
continuing execution I found where the two had all the same valid data members and when finished the $==$ function I
discovered that the returned value was false, which would explain why deleteItem never found the right value.

At this point I was feeling pretty good about having narrowed down the problem to a calculation issue, but why was I
getting invalid values for eccentricity and theta for some of the regions?  To determine this I set a breakpoint in the
setData function of RegionType and recreated the error yet again.  To my surprise the first region had some invalid
values, but I also noticed that the value for lambdaMin was zero; I thought I recalled the eccentricity requiring
dividing by lambdaMin, so I checked it out.  I verified that this was the indeed true, so I decided to find why
lambdaMin was zero.  After using similar techniques I discovered that lambdaMin was zero because the central moment was
returning zero with non-zero parameters because the centroids were equal to the location of the pixel.  This calculation
was correct and I received the same value when I did the calculations by hand, so the problem wasn't actually a problem
with the code, it was a problem with the function (I found that this is true for any shape that is 1 pixel wide or
long).  As for the value of theta, I discovered that if the shape is a single pixel or has an orientation of exactly
parallel to the x-axis then the central moment with parameters 1, 1 was returning zero.  I added exceptions for both of
these (making the orientation one-hundred eighty degrees if in this case, although zero would have been equally as
correct).  This single bug took nearly two hours to track down, but after discovering the cause I feel much more
experienced with GDB.



\section{What was Learned}

\section{Division of Labor}

\section{Extra Credit}

\end{document}
